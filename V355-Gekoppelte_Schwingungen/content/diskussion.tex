\section{Diskussion}
\label{sec:Diskussion}

\subsection{Frequenzverhältnisse bei einer Schwebung}
\label{sec:Schwebung}

Auffällig sind die starken Abweichungen zu den Erwartungswerten beim $4.74nF$ sowie beim $8.18nF$ Koppelkondensator, wobei letzterer noch durch den veranschlagten Fehler von $\pm 1$ noch akzeptiert werden kann. Die verbleibenden Werte entsprechen vollständig den Erwartungen. Weiterhin ist in Betracht zu ziehen, dass die Abstimmung beider Schwingkreise nur begrenzt exakt ist.

\subsection{Messung von $v_+$ und $v_-$ mittels Lissajous-Figuren}
\label{sec:messung}

Im niederkapazitiven Berreich des Koppelkondensator fallen die Messungen von $v_-$ sehr präzise aus, während bei $v_+$ in diesem Berreich die größten Abweichungen auftreten. Bei höherkapazitiven Koppelkondensatoren nimmt die Exaktheit der Messwerte für $v_+$ stark zu, dafür weicht $v_-$ zunehmend stark von den erwarteten Werten ab. Dies kann durch die hier nicht betrachteten, weil unbekannten Ungenauigkeiten der übrigen Bauteile liegen. Ein weiteres Problem besteht in der Erdung der Schaltung 2, welche nicht zuverlässig funktioniert und daher die Schärfe der Lissajous-Figuren begrenzt.

\subsection{Messung von $v_+$ und $v_-$ mittels Sweap-Verfahren}
\label{sec:sweap}

Dieses Verfahren war deutlich schneller als die Erfassung der Grundschwingungen mittels Lissajous-Figuren, wenn die nötigen Einstellungen am sweap-Generator getroffen sind. Es eignet sich somit gut, wenn für viele Kondensatoren die Grundschwingungen grob bestimmt werden sollen. Abb. \ref{fig:sweap} lässt sich jedoch klar entnehmen, dass die Messreihe stärker von den Theoriewerten abweicht als die in Abb. \ref{fig:frequenzen} dargestellte erste Messreihe. Augenscheinlich fallen alle Messwerte geringer aus und ihre Präzision nimmt mit steigender Kapazität ab. Letzteres liegt daran, dass der am Oszilloskop auftretende Graph mit stärkerer Kopplung zunehmend gestaucht wird und so die Ablesegenauigkeit rapide abnimmt. Dies kann durch Nachjustierung des Generators behoben werden, hierdurch entfällt jedoch der zeitliche Vorteil dieses Verfahrens.

