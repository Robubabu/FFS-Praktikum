\section{Diskussion}
\label{sec:Diskussion}

\subsection{Frequenzverhältnisse bei einer Schwebung}
\label{sec:Schwebung}

Auffällig sind die starken Abweichungen zu den Erwartungswerten beim $4.74nF$ von $1.78 \pm 1.19$, sowie beim $8.18nF$ Koppelkondensator von $1.29 \pm 1.32$, wobei letzterer noch durch den veranschlagten Fehler von $\pm 1$ beim Ablesen der Peaks noch akzeptiert werden kann. Die verbleibenden Werte entsprechen den Erwartungen in sodern, als dass in Abb. \ref{fig:verhaeltnisse} klar erkennbar ist, dass die Abweichungen innerhalb der Toleranzen liegen. Weiterhin ist in Betracht zu ziehen, dass die Abstimmung beider Schwingkreise nur begrenzt exakt ist.

\subsection{Messung von $v_+$ und $v_-$ mittels Lissajous-Figuren}
\label{sec:messung}

Im niederkapazitiven Bereich des Koppelkondensators fallen die Messungen von $v_-$ ($\Delta v \approx 0.12kHz \pm 0.05kHz$) sehr präzise aus, während bei $v_+$ in diesem Berreich die größten Abweichungen ($0.91kHz$) auftreten. Bei höherkapazitiven Koppelkondensatoren nimmt die Exaktheit der Messwerte für $v_+$ stark zu ($\Delta v \approx 0.17$), dafür weicht $v_-$ zunehmend stark von den erwarteten Werten ab ($\Delta v \approx 0.8kHz$). Dies kann durch die hier nicht betrachteten, weil unbekannten Ungenauigkeiten der übrigen Bauteile liegen. Ein weiteres Problem besteht in der Erdung der Schaltung 2, welche nicht zuverlässig funktioniert und daher die Schärfe der Lissajous-Figuren begrenzt. Dennoch konnte eine konstante Spannung $v_+$ sowie ein \eqref{eqn:nu-} entsprechender Verlauf beobachtet werden.

\subsection{Messung von $v_+$ und $v_-$ mittels Sweep-Verfahren}
\label{sec:sweep}

Dieses Verfahren war deutlich schneller als die Erfassung der Grundschwingungen mittels Lissajous-Figuren, wenn die nötigen Einstellungen am sweep-Generator getroffen sind. Es eignet sich somit gut, wenn für viele Kondensatoren die Grundschwingungen grob bestimmt werden sollen. Abb. \ref{fig:sweep} lässt sich jedoch klar entnehmen, dass die Messreihe für $v_-$ stärker von den Theoriewerten abweicht ($\Delta v_- \approx 1.6kHz$, $\Delta v_+ \approx 1.28 kHz$) als die in Abb. \ref{fig:frequenzen} dargestellte erste Messreihe ($\Delta v_- \approx 0.8kHz$, $\Delta v_+ \approx 0.9kHz$), während die Abweichungen von $v_+$ ähnlich präzise ausfallen. Augenscheinlich fallen alle Messwerte geringer aus und ihre Präzision nimmt mit steigender Kapazität ab. Letzteres liegt daran, dass die Differenz zwischen $v_+$ und $v_-$ mit stärkerer Kopplung abnimmt, wodurch der am Oszilloskop auftretende Graph beständig schmaler wird und so die Ablesegenauigkeit rapide abnimmt. Dies kann durch Nachjustierung des Generators behoben werden, hierdurch entfällt jedoch der zeitliche Vorteil dieses Verfahrens.
