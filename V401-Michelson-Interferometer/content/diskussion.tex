\section{Diskussion}
\label{sec:Diskussion}
\subsection{Messung der Wellenlänge}


Die gemessene Wellenlänge $\lambda = \SI{610 \pm 40}{\nano \meter}$ (Ungenauigkeit ca. $6.6\%$)deckt sich  mit der tatsächlichen Wellenlänge des Lasers $\lambda_{0} = \SI{635}{\nano\meter}$ sehr gut, trotz der hohen Schwankung der Messerwerte ($(402-820)\;\si{\nano \meter}$), welche durch die enorme Empfindlichkeit des Versuchs bedingt ist, da die Abweichung zwischen Erwartungswert und experimentellem Wert mit $\Delta \lambda = \SI{25}{\nano \meter}$ ($3.9\%$) weniger als die Ungenauigkeit von $\lambda$ beträgt. Die relative Abweichung beträgt ca. $2$ bis $10\%$ (mini- bzw. maximale Abweichung). Leider waren die Erschütterungen der Apparatur nur zu mindern, nicht zu verhindern, da am Nebentisch zeitgleich experimentiert wurde. Eventuell hätte eine andere Justierung oder eine andere Stellgeschwindigkeit genauere Ergebnisse erzielt, da ein leichtes Flackern des Interferenzmusters zu beobachten war, welches die Zählung beeinflusste. Ein schärferes, dafür allerdings kleineres Interferenzmuster könnte dementsprechent bessere Ergebnisse liefern.


\subsection{Brechungsindex}

Mit den experimentellen Werten von $n_{air,0}-1 = (2.7 \pm 0.1)\cdot 10^{-4}$ und $n_{CO_2,0}-1 = (4.3 \pm 0.1)\cdot 10^{-4}$ erhält man beim Vergleich mit den Theoriewerten für die Brechung von Licht der Wellenlänge $\SI{635}{\nano\meter}$ ($n_{air,lit} = 1.00028$ und $n_{CO_2,lit} = 1.00045$) Abweichungen von $\delta \left(n_{air}-1\right) = (0.1 \pm 0.1)\cdot 10^{-4}$ bzw. $\delta \left(n_{CO_2}-1\right) = (0.2 \pm 0.1)\cdot 10^{-4}$. Da die Abweichung bei Luft der Messungenauigkeit entspricht, wird dieser Wert als im Rahmen der Möglichkeiten der Apparatur und der Experimentatoren als exakt bestimmt angenommen. Die relative Ungenauigkeit beim Kohlendioxid beträgt $\approx 4.2\%$, was ein Verhältnismäßig gutes Ergebnis darstellt.
