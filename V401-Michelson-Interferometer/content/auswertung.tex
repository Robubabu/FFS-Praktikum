\section{Auswertung}
\label{sec:Auswertung}
Die Auswertung aller Daten wurde mittels Python \ref{scipy} \ref{numpy} durchgeführt.

\subsection{Messung der Wellenlänge}

Alle gemessenen Werte lassen sich Tab. \ref{tab:Wavelength} entnehmen. Dieser Tabelle sind auch die daraus hervorgehenden Wellenlängen zu entnehmen, welche sich aus Formel

%%%%%%%%%%%%%%%%%%%%%%%%%%%%%%%%%%%%%%%%%%%%%%%%%%%%%%%%%%%
\newpage
  WICHTIG!!! WICHTIG!!!! REFERENZ WELLENLÄNGE ERGÄNZEN!!!!!!!
  \newpage
  %%%%%%%%%%%%%%%%%%%%%%%%%%%%%%%%%%%%%%%%%%%%%%%%%%%%%%%%

ergeben.
Der Mittelwert der Wellenlänge beträgt somit
\begin{equation*}
  \lambda = \SI{610 \pm 40}{\nano \meter}.
\end{equation*}

\begin{table}
  \centering
  \caption{Gemessene Größen und daraus resultierende Wellenlängen.}
  \label{tab:Wavelength}
  \begin{tabular}{S S S}
    \toprule
    $Counts$ & $x/\si{\milli\meter}$ & $\lambda/\si{\nano\meter}$ \\
    \midrule
    1003 & 1.61 & 640\\
    993 & 1.49 & 598\\
    1014 & 1.58 & 621\\
    1151 & 1.67 & 578\\
    1025 & 1.47 & 572\\
    1159 & 1.24 & 427\\
    1045 & 2.15 & 820\\
    1026 & 1.54 & 598\\
    1011 & 1.02 & 402\\
    1120 & 2.24 & 797\\
    \bottomrule
  \end{tabular}
\end{table}

\subsubsection{Bestimmung des Brechnungsindex}

\begin{align*}
  n_{air,0}-1 &= 2.7e-4 \pm 0.1e-4 \\
  n_{CO_2,0}-1 &= 4.3e-4 \pm 0.1e-4
\end{align*}
