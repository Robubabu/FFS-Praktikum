\section{Auswertung}
\label{sec:Auswertung}
Die Auswertung aller Daten wurde mittels Python \cite{scipy} \cite{numpy} durchgeführt.

\subsection{Messung der Wellenlänge}

Alle gemessenen Werte lassen sich Tab. \ref{tab:Wavelength} entnehmen. Dieser Tabelle sind auch die daraus hervorgehenden Wellenlängen zu entnehmen, welche sich aus Formel

%%%%%%%%%%%%%%%%%%%%%%%%%%%%%%%%%%%%%%%%%%%%%%%%%%%%%%%%%%%
\newpage
  WICHTIG!!! WICHTIG!!!! REFERENZ WELLENLÄNGE ERGÄNZEN!!!!!!!
  \newpage
  %%%%%%%%%%%%%%%%%%%%%%%%%%%%%%%%%%%%%%%%%%%%%%%%%%%%%%%%

ergeben.
Der Mittelwert der Wellenlänge beträgt somit
\begin{equation*}
  \lambda = \SI{610 \pm 40}{\nano \meter}.
\end{equation*}

\begin{table}
  \centering
  \caption{Gemessene Größen und daraus resultierende Wellenlängen.}
  \label{tab:Wavelength}
  \begin{tabular}{S S S}
    \toprule
    $Counts$ & $x/\si{\milli\meter}$ & $\lambda/\si{\nano\meter}$ \\
    \midrule
    1003 & 1.61 & 640\\
    993 & 1.49 & 598\\
    1014 & 1.58 & 621\\
    1151 & 1.67 & 578\\
    1025 & 1.47 & 572\\
    1159 & 1.24 & 427\\
    1045 & 2.15 & 820\\
    1026 & 1.54 & 598\\
    1011 & 1.02 & 402\\
    1120 & 2.24 & 797\\
    \bottomrule
  \end{tabular}
\end{table}

\subsection{Bestimmung des Brechungsindex}


Der Brechungsindex von Luft ($n_{air,0}$) bzw. Kohlenstoffdioxid ($n_{CO_2,0}$) ergibt sich über
\newpage
      WICHTIG!!!!! REFERENZ EINFÜGEN!!!!!!
\newpage

mit den Werten aus Tab. \ref{tab:N_tab} zu

\begin{align*}
  n_{air,0}-1 &= 2.7e-4 \pm 0.1e-4 \\
  n_{CO_2,0}-1 &= 4.3e-4 \pm 0.1e-4.
\end{align*}

\begin{table}
  \centering
  \caption{Messwerte zur Bestimmung der Brechungsindizes, sowie daraus ergebene $\Delta n \pm \delta \Delta n$.}
  \label{tab:N_tab}
  \sisetup{round-mode=places}
  \begin{tabular}{S|S[round-precision = 1]|S[round-precision = 1]|S|S|S[round-precision = 1]|S[round-precision = 1]}
    \toprule
    $Count_{Air}$ &  $\Delta n_{air}$ &  $\delta \Delta n_{air}$ & $p/\si{\bar}$ & $Count_{CO_2}$ & $\Delta n_{CO_2}$ & $\delta \Delta n_{CO_2}$ \\
    \midrule
    32 & 1.937235837277207298e-04 & 1.347768911995184540e-05 & -0.88 & 56 & 3.390162715235112297e-04 & 2.358595595991572818e-05\\
    32 & 1.937235837277207298e-04 & 1.347768911995184540e-05 & -0.80 & 44 & 2.663699276256159933e-04 & 1.853182253993378764e-05\\
    32 & 1.937235837277207298e-04 & 1.347768911995184540e-05 & -0.80 & 50 &  3.026930995745636386e-04 & 2.105888924992475452e-05\\
    33 & 1.997774457192120221e-04 & 1.389886690495033988e-05 & -0.80 & 50 & 3.026930995745636386e-04 & 2.105888924992475452e-05\\
    33 & 1.997774457192120221e-04 & 1.389886690495033988e-05 & -0.80 & 56 & 3.390162715235112297e-04 & 2.358595595991572818e-05\\
    33 & 1.997774457192120221e-04 & 1.389886690495033988e-05 & -0.80 & 57 & 3.450701335150025491e-04 & 2.400713374491422435e-05\\
    33 & 1.997774457192120221e-04 & 1.389886690495033988e-05 & -0.80 & 50 & 3.026930995745636386e-04 & 2.105888924992475452e-05\\
    33 & 1.997774457192120221e-04 & 1.389886690495033988e-05 & -0.80 & 49 & 2.966392375830723734e-04 & 2.063771146492626173e-05\\
    33 & 1.997774457192120221e-04 & 1.389886690495033988e-05 & -0.80 & 55 & 3.329624095320200187e-04 & 2.316477817491723539e-05\\
    33 & 1.997774457192120221e-04 & 1.389886690495033988e-05 & -0.80 & 52 & 3.148008235575461690e-04 & 2.190124481992174687e-05\\

    \bottomrule
  \end{tabular}
\end{table}
