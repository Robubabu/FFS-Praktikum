\section{Auswertung}
\label{sec:Auswertung}
Die Auswertung aller Daten wurde mittels Python \cite{scipy} \cite{numpy} durchgeführt.

\subsection{Messung der Wellenlänge}

Alle gemessenen Werte lassen sich Tab. \ref{tab:Wavelength} entnehmen. Dieser Tabelle sind auch die daraus hervorgehenden Wellenlängen zu entnehmen, welche sich aus Formel \eqref{eqn:dd} ergeben.
Der Mittelwert der Wellenlänge beträgt somit
\begin{equation*}
  \lambda = \SI{610 \pm 40}{\nano \meter}.
\end{equation*}

\begin{table}
  \centering
  \caption{Gemessene Größen und daraus resultierende Wellenlängen.}
  \label{tab:Wavelength}
  \begin{tabular}{S S S}
    \toprule
    $Counts$ & $x/\si{\milli\meter}$ & $\lambda/\si{\nano\meter}$ \\
    \midrule
    1003 & 1.61 & 640\\
    993 & 1.49 & 598\\
    1014 & 1.58 & 621\\
    1151 & 1.67 & 578\\
    1025 & 1.47 & 572\\
    1159 & 1.24 & 427\\
    1045 & 2.15 & 820\\
    1026 & 1.54 & 598\\
    1011 & 1.02 & 402\\
    1120 & 2.24 & 797\\
    \bottomrule
  \end{tabular}
\end{table}

\subsection{Bestimmung des Brechungsindex}


Der Brechungsindex von Luft ($n_{air,0}$) bzw. Kohlenstoffdioxid ($n_{CO_2,0}$) ergibt sich über \eqref{eqn:n} mit den Werten aus Tab. \ref{tab:N_tab} zu

\begin{align*}
  n_{air,0}-1 &= (2.7 \pm 0.1)\cdot 10^{-4} \\
  n_{CO_2,0}-1 &= (4.3 \pm 0.1)\cdot 10^{-4}.
\end{align*}

\begin{table}
  \centering
  \caption{Messwerte zur Bestimmung der Brechungsindizes, sowie daraus ergebene $\Delta  n \pm \delta \Delta n$.}
  \label{tab:N_tab}
  \sisetup{round-mode=places}
  \begin{tabular}{S|S[round-precision = 1]|S[round-precision = 1]|S|S|S[round-precision = 1]|S[round-precision = 0]}
    \toprule
    $Count_{air}$ & $\Delta n_{air}$ & $\delta n_{air}$ & $\Delta p/\si{\bar}$ & $\Delta n_{CO2}$ & $\delta  n_{CO2}$ \\
    \midrule
    32 & 1.93723583727720729804 & 0.1347768911995184540   & -0.88 & 56 & 3.39016271523511229704 & 0.235859559599157281805\\
    32 & 1.93723583727720729804 & 0.134776891199518454005 & -0.80 & 44 & 2.66369927625615993304 & 0.185318225399337876405\\
    32 & 1.93723583727720729804 & 0.134776891199518454005 & -0.80 & 50 &  3.0269309957456363864 & 0.210588892499247545205\\
    33 & 1.99777445719212022104 & 0.138988669049503398805 & -0.80 & 50 & 3.02693099574563638604 & 0.210588892499247545205\\
    33 & 1.99777445719212022104 & 0.138988669049503398805 & -0.80 & 56 & 3.39016271523511229704 & 0.235859559599157281805\\
    33 & 1.99777445719212022104 & 0.138988669049505       & -0.80 & 57 & 3.45070133515002549104 & 0.240071337449142243505\\
    33 & 1.99777445719212022104 & 0.138988669049505       & -0.80 & 50 & 3.02693099574563638604 & 0.210588892499247545205\\
    33 & 1.99777445719212022104 & 0.138988669             & -0.80 & 49 & 2.96639237583072373404 & 0.206377114649262617305\\
    33 & 1.99777445719212022104 & 0.13898866904950335     & -0.80 & 55 & 3.32962409532020018704 & 0.231647781749172353905\\
    33 & 1.99777445719212022104 & 0.1                     & -0.80 & 52 & 3.14800823557546169000 & 0.219012448199217468705\\

    \bottomrule
  \end{tabular}
\end{table}
