\section{Diskussion}
\label{sec:Diskussion}
Bei der gegebenen Apparatur lagen starke Schwankungen des Photostroms, hervorgerufen durch durch Flackern der Lampe, vor, was sich vor allem in Abb. \ref{fig:plot} bei orangenem Licht niederschlägt. Dieses Flackern könnte auch der Grund dafür sein, dass der asymptotische Photostrom in diesem Versuch nicht aufgezeigt werden konnte, welcher sich eigentlich daraus ergibt, dass ab einer gewissen Spannung alle ausgelösten Elektronen, die Anode erreichen.  Da diese bei Zimmertemperatur verdampft, treten Stöße zwischen den ausgelösten Elektronen und eben jenen Kathodenatomen auf, was die Elektronen bremst. Zudem gibt es gemäß der Fermi-Dirac-Verteilung Elektronen, welche eine höhere Energie haben als der Durchschnitt und dazu mit der Gegenspannung entgegengesetzten Geschwindigkeit austreten, oder schlicht die Anoder verfehlen, ist es kaum möglich alle Elektronen zu detektieren. Dieser Stromverlauf ist folglich kein Widerspruch zum Ohmschen Gesetz, schon allein da kein Ohmscher Widerstand betrachtet wird, besagtes Gesetz also keinerlei Relevanz hat. Um den Maximalwert des Stromes tatsächlich zu erreichen, müsste die Anode also die Kathode vollkommen umschließen, was jedoch das Licht abschirmt und den Photoeffekt somit unterbindet.
Da die Anode und die Kathode aus unterschiedlichem Material bestehen, ergibt sich zwischen ihnen durch ihre unterschiedlichen Fermi-Niveaus ein Kontaktpotential, gegen das die Elektronen anlaufen müssen, d.h. die effektive Spannung ist geringer als die eingestellte.
Der experimentelle Wert für $h/e$ mit $h=\SI{4.6 \pm 0.9}{\femto \volt \second}$ liegt trotz der Schwankungen im erwarteten Bereich, allerdings fällt auf, dass die Ausgleichsgerade an 3 von 4 Fehlerbalken vorbei läuft, der Fit also sehr ungenau ist. Zudem  stehen nur vier Werte zur Verfügung, was auch bei einem lediglich linearen Fit sehr wenig ist.
