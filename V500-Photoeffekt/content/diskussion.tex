\section{Diskussion}
\label{sec:Diskussion}
Bei der gegebenen Apparatur lagen starke Schwankungen des Photostroms, hervorgerufen durch durch Flackern der Lampe, vor, was sich vor allem in Abb. \ref{fig:plot} bei orangenem Licht niederschlägt. Dieses Flackern könnte auch der Grund dafür sein, dass der asymptotische Photostrom in diesem Versuch nicht aufgezeigt werden konnte. Des Weiteren verdampft die Kathode bei Zimmertemperatur, weshalb Stöße zwischen den ausgelösten Elektronen und eben jenen Kathodenatomen auftreten können.
Dennoch liegt der experimentelle Wert für $h$ mit $h=\SI{4.6 \pm 0.9}{\femto \electronvolt \second}$ im erwarteten Berreich, allerdings fällt auf, dass die Ausgleichsgerade an 3 von 4 Fehlerbalken vorbei läuft, der Fit also sehr ungenau ist. Zudem  stehen nur vier Werte zur Verfügung, was auch bei einem lediglich linearen Fit sehr wenig ist.
