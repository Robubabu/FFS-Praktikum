\section{Versuchsziel}
Im Versuch soll verifiziert werden, dass
\begin{itemize}
  \item die Anzahl der vom Photoeffekt ausgelösten Elektronen
  proportional zur Lichtntensität ist.
  \item die Energie der Photoelektronen proportional zur Lichtfrequenz ist.
  \item eine Grenzfrequenz existiert für die der Photoeffekt nicht auftritt.
\end{itemize}
\section{Theorie}
\label{sec:Theorie}
Der Photoeffekt bezeichnet das Phänomen, dass sich Elektronen unter Lichteinstrahlung
aus Metalloberflächen lösen. Um diesen Effekt zu untersuchen wir nun eine Photokathode
im Vakuum mit monochromatischem Licht bestrahlt. Ihr gegenüber ist eine
Auffängerelektrode angebracht, welche die freigesetzten Elektronen anzieht. Die
Elektroden sind mit einander durch ein Strommessgerät verbunden. Somit ergibt
sich der Zusammenhang
\begin{equation}
  \symup{h}\nu = E_{kin} + A_k
  \label{eqn:Z}
\end{equation}
dabei bezeichnet $ \symup{h} \nu $ die Energie der Photonen (also Planksches
Wirkungsquantum mal Frequenz), $E_{kin} $ die
kinetische Energie der Elektronen und $A_k$ die Austrittsarbeit die gleistet
werden muss bevor das Elektron austretten kann.
