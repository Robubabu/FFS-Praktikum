\section{Versuchsziel}
Im Versuch soll
\begin{itemize}
  \item die Grenzspannung bestimmt werden, ab der ein Photostrom auftritt.
  \item die Charakteristik des Photostromes bestimmt werden.
  \item $e/\symup{h}$ bestimmt werden.
  \item die Auslösearbeit $A_k$ des Kathodenmaterials bestimmen.
\end{itemize}
\section{Theorie}
\label{sec:Theorie}
Der Photoeffekt bezeichnet das Phänomen, dass sich Elektronen unter Lichteinstrahlung
aus Metalloberflächen lösen. Um diesen Effekt zu untersuchen, wird nun eine Photokathode
im Vakuum mit monochromatischem Licht bestrahlt. Ihr gegenüber ist eine
Auffängerelektrode angebracht, welche die freigesetzten Elektronen anzieht. Die
Elektroden sind mit einander durch ein Strommessgerät verbunden. Für die Energie $E_{\symup{kin}}$ der ausgelösten Elektronen gilt der zur Frequenz $\nu$ lineare Zusammenhang
\begin{equation}
  E_{\symup{kin}} = \symup{h} \nu - A_k.
  \label{eqn:Z}
\end{equation}
$\symup{h}$ bezeichnet hierbei das Plancksche Wirkungsquantum. $\symup{h} \nu$ ist die Energie, welche das Licht an einzelne Elektronen abgibt, also die Energie eines Photons. Die Photonentheorie des Lichtes sagt aus, dass Licht in einzelnen Photonen gequantelt existiert und nur seine gesamte Energie oder keine Energie abgeben kann. In diesem Bild ist die Intensität des Lichtes nicht weiter als die Zahl der Photonen pro Fläche und Zeit, weshalb bei höherer Intensität des bei der Untersuchung des Photostroms verwendeten Lichtes lediglich mehr Elektronen austreten und somit der Photostrom zwischen den Elektroden des Aufbaus steigt.
