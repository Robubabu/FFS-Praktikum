\section{Durchführung}
\label{sec:Durchführung}
Zur Untersuchung des Photoeffektes wird eine Photokathode
im Vakuum mit monochromatischem Licht bestrahlt. Ihr gegenüber ist eine
Auffängerelektrode angebracht, welche die freigesetzten Elektronen anzieht. Die
Elektroden sind mit einander durch ein Strommessgerät verbunden. Die Kathode besteht hier aus einer auf der
Innenseite aufgedampfte Metall- oder Legierungsschicht. Die Anode wird durch einen
Drahtring realisiert der mit geringem Abstand zur Kathode angebracht wurde.
Die Abbildung \ref{fig:PZ} zeigt eine schematische Abbildung der Photozelle.
Die Abbildung \ref{fig:VAS} zeigt dagegen den schematischen Versuchsaufbau,
der es ermöglicht, die Energie der Photoelektronen mit der Gegenfeldmethode zu
bestimmen. Hierbei wird die Energie der Elektronen über die elektrostatische Spannung bestimmt, die sie überwinden können. Es gilt damit
\begin{equation}
  E_{\symup{kin}} = \symup{e} \cdot U_{\symup{g}}
  \label{eqn:hv}
\end{equation}
zu erweitern. Dabei bezeichnet $\symup{e}$ die Elementarladung und $U_\symup{g}$ die
Gegenspannung, bei der der Photostrom gerade verschwindet. Dieser Zusammenhang lässt sich mit \eqref{eqn:Z} gleichsetzen.
\begin{figure}
  \centering
    \begin{subfigure}{0.48\textwidth}
      \centering
      \includegraphics[height=5cm]{logos/Photozelle.png}
      \caption{Die Photozelle - evakuierte Kathode und Anode zur Untersuchung des Photoeffekts \cite{Anleitung}.}
      \label{fig:PZ}
    \end{subfigure}
    \begin{subfigure}{0.48\textwidth}
      \centering
      \includegraphics[height=5cm]{logos/VASchaltung.png}
      \caption{Versuchsaufbau zur Messung des Stroms und Bestimmung der Energie der Elektronen durch die Gegenfeldmethode \cite{Anleitung}. }
      \label{fig:VAS}
    \end{subfigure}
  \caption{Schematische Abbildungen}
  \label{fig:PZS}
\end{figure}
Zur Erzeugung monochromatischen Lichtes wird der Versuch wie in Abbildung
\ref{fig:VAO} aufgebaut. Nun müssen erst die Linsen und der Spalt so eingestellt
werden, dass sich das Licht im Prisma bricht und alle Spektrallinien des Lichtes
getrennt, scharf und mit möglichst großer Intensität vor der Photozelle zu beobachten
sind. Dann können die jeweiligen Spektrallinen durch den Schwenkarm auf die
Photozelle gerichtet werden.

\begin{figure}
  \centering
  \includegraphics[height=3cm]{logos/VAOptisch.png}
  \caption{Schemtaischer Versuchsaufbau zur Erzeugung des monochromatischen Lichtes \cite{Anleitung}.}
  \label{fig:VAO}
\end{figure}

\paragraph{Messporgramm}
Da hier eine Hg-Dampflampe benutzt wird, wird für alle Spektrallinie der
Photostrom abhängig von der Gegenspannung aufgenommen, außer für die rote Spektrallinie, da diese nicht zum Hg-Spektrum gehört. Aus ihrer Existenz lässt sich das Eintreten von Fremdgas in die Dampflampe ableiten.
Die Spannung wird dabei von $ \SI{-2}{\volt} \leq U \leq \ \SI{2}{\volt}$ variiert.
Dann wird nur für die gelbe Spektrallinie (\SI{578}{\nano\meter}) der Photostrom
abhängig von der Spannung bestimmt. Diesesmal soll aber die Spannung von
\SI{-20}{\volt} bis \SI{20}{\volt} variiert werden, um den charakteristischen Verlauf des Photostroms der verwendeten Diode zu untersuchen.
