\section{Diskussion}
\label{sec:Diskussion}
\subsection{Vergleich der Abmessungen bestimmt durch Ascan und Bscan}
Zum Vergleich wurden die relativen Abweichungen mit der Formel
\begin{equation*}
  \tilde{x} = \frac{\lvert x - y\rvert}{\lvert y \rvert} \cdot 100 \%
\end{equation*}
bestimmt. Hier bei wäre $x$ die Werte vom Ascan und $y$ vom Bscan. Die Werte
dazu sind in der Tabelle \ref{tab:RF} dargstellt. Es ist unschwer zu erkennen,
dass die Abweichung sehr stark ist und somit nicht mehr durch allgemeine Messungenauigkeit
erklärt werden.
 Wie zu erkennen ist liegen die Werte des
Bscans schon ein Faktor 10 in der Größenordnung von den Werten des Ascan entfernd. Deshalb
wird ein Notationsfehler bei Aufnahme der Daten oder ein Umrechnungsfehler vermutet.

\begin{table}
  \centering
    \caption{Relativen Abweichung der Werte vom Ascan zum Bscan}
    \sisetup{round-mode=places, round-precision= 2}
    \resizebox{\textwidth}{!}{
    \begin{tabular}{S[scientific-notation = fixed , fixed-exponent = 0, round-precision=0] S@{$\quad\pm{}$} S S@{$\quad\pm{}$}  S S[scientific-notation = fixed , fixed-exponent = 0, round-precision=0]@{$\quad\pm{}$}  S}
      \toprule
      $ \text{Fehlstellen} $& \multicolumn{2}{c}{$F_{A} \pm \Delta F_{A}$ /\si{\meter}} &\multicolumn{2}{c}{$F_{B} \pm \Delta F_{B}$ /\si{\meter}}& \multicolumn{2}{c}{$\tilde{F} \pm \Delta \tilde{F}$ /\si{\percent}}  \\
      \midrule
      1.100000000000000000e+01 & 9.455499999999991689e-03 & 0.000000000000000000e+00 & 4.207899999999999141e-02 & 0.000000000000000000e+00 & 7.752917132061124050e+01 & 0.000000000000000000e+00\\
      9.000000000000000000e+00 & 2.493999999999996220e-03 & 0.000000000000000000e+00 & 2.597199999999998815e-02 & 0.000000000000000000e+00 & 9.039735099337748636e+01 & 0.000000000000000000e+00\\
      8.000000000000000000e+00 & 2.084500000000003017e-03 & 0.000000000000000000e+00 & 1.000149999999999650e-02 & 0.000000000000000000e+00 & 7.915812628105780391e+01 & 0.000000000000000000e+00\\
      7.000000000000000000e+00 & 2.220999999999986874e-03 & 0.000000000000000000e+00 & 5.969000000000002082e-03 & 0.000000000000000000e+00 & 6.279108728430246344e+01 & 0.000000000000000000e+00\\
      6.000000000000000000e+00 & 2.357499999999984608e-03 & 0.000000000000000000e+00 & 2.057449999999999557e-02 & 0.000000000000000000e+00 & 8.854164135215928866e+01 & 0.000000000000000000e+00\\
      5.000000000000000000e+00 & 3.312999999999982625e-03 & 0.000000000000000000e+00 & 3.504350000000001908e-02 & 0.000000000000000000e+00 & 9.054603564141714855e+01 & 0.000000000000000000e+00\\
      4.000000000000000000e+00 & 4.814499999999999336e-03 & 0.000000000000000000e+00 & 5.033150000000001512e-02 & 0.000000000000000000e+00 & 9.043441979674756226e+01 & 0.000000000000000000e+00\\
      3.000000000000000000e+00 & 5.633499999999999619e-03 & 0.000000000000000000e+00 & 3.886550000000001115e-02 & 0.000000000000000000e+00 & 8.550513951962538783e+01 & 0.000000000000000000e+00\\
      2.000000000000000000e+00 & 4.465000000000024505e-04 & 1.365000000000032908e-04 & 4.964900000000001257e-02 & 1.365000000000032908e-04 & 9.910068682148683195e+01 & 2.774024904603794028e-01\\
      1.000000000000000000e+00 & 5.830000000000001847e-04 & 1.365000000000032908e-04 & 4.582699999999999274e-02 & 1.365000000000032908e-04 & 9.872782420843607554e+01 & 3.016486349869904049e-01\\
      \bottomrule
    \end{tabular}
    }
    \label{tab:RF}
  \end{table}
\subsection{Zur Untersuchung des Herzmodelles}
Die Untersuchung ergab ein Schlagvolumen von \SI{30}{\milli\liter}. Bei einem
richtigen menschlichen Herzen liegt dieses bei ca. 70 bis \SI{100}{\milli\liter}
\cite{wiki}. Diese Abweichung ist aber ehr damit zu erklären, dass das Modell
nicht sehr naturgetreu ist. 
