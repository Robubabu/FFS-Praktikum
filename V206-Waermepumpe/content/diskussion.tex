\section{Diskussion}
\label{sec:Diskussion}
\paragraph{Temperaturverläufe}
Bei den Differentialkoeffizienten fällt der relative Fehler von $\approx 50 \%$ auf, welcher daher rührt, dass bereits der Fit in allen Parametern fehlerbehaftet ist und sich dies durch seine Ableitung weiter verstärkt. Der Fit selbst ist (wie aus Abb. \ref{fig:temp} zu entnehmen) relativ präzise.

\paragraph{Güte der Apparatur}
Die Fehler der Güteziffern ist enorm, die Abweichung von Theoriewert und Realwert liegt in einer ganzen Größenordnung. Dies ergibt sich durch viele (teils unvermeidbare) Mängel der Apparatur, wie zB. eine Erwärmung durch Betrieb des Kondensators, nicht vollständig dichte Behälter und begrenzte Isolierung der Leitungen.

\paragraph{Massendurchsatz}
Auch hier fallen die relativen Fehler wieder um etwa $50\%$ aus, was daran liegt, dass $\frac{\symup{d}m}{\symup{d}t}$ unmittelbar mit $\frac{\symup{d}T}{\symup{d}t}$ zusammenhängt.

\paragraph{Kompressorleistung}
An der mechanischen Kompressorleistung erkennt man gut, dass ein großer Teil der in den Kompressor gespeißten Leistung verloren geht $(\Delta N \approx 100 \si{\watt})$ und vor allem als Wärme an die Umgebung und den restlichen Aufbau abgegeben wird.
