\section{Durchführung}
\label{sec:Durchführung}
Ein Messdruchgang wird in den folgenden Punkten dargestellt, dabei ist
darauf zu achten, dass die Auslenkungen die klein Winkelnährung $\symup{sin}\phi \approx \phi$
einhalten.
\begin{itemize}
  \item Bestimmt werden $T_1 $ und $T_2$ der beiden ungekoppelten Pendel.
  Dazu werden immer fünf Schwingungen gemessen und das zehnmal.
  Dabei soll erreicht werden, dass die Pendellängen so eingestellt werden,
  dass im Rahmen der Messungenauigkeiten $T_1 = T_2$ gilt. Ist dies nicht der
  Fall muss nachjustiert werden, indem die Masse auf dem Pendel verschoben wird.
  Zur Überprüfung muss dieser Schritt danach wiederholt werden.
  \item Jetzt werden die Pendel gekoppelt.
  \item Dann wird die Schwingungsdauer $T_+$ für die gleichsphasige Schwingung
  bestimmt. Wieder werden fünf Schwingungen zehnmal gemessen.
  \item Darauf wird die Schwingungsdauer $T_-$ für die gegenphasigen Schwingung
  bestimmt. Wieder werde fünf Schwingungen zehnmal gemessen.
  \item Nun wird zehnmal die Schwingungsdauer $T$ und die Schwebungsdauer $T_S$
  für die gekoppelte Schwingung gemessen.
  \item Dann wird der Messdurchgang für mindestens eine weitere Pendellänge
  wiederholt.
\end{itemize}
