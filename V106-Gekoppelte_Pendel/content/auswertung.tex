\section{Auswertung}
\label{sec:Auswertung}

\paragraph{Verwendete Pendel}
Im ersten Durchlauf werden die Pendel auf $l_{1,l} =(58.233 \pm 0.047)\si{\centi\meter}$ und $l_{1,r} =(58.433 \pm 0.047)\si{\centi\meter}$ eingestellt, sodass sich Schwingungsdauern von $T_{1,l} =(1.479 \pm 0.020)\si{\second}$ und $T_{1,r} = (1.466 \pm 0.014)\si{\second}$ einstellen.
Beim zweiten Durchlauf ergeben sich $T_{2,l}=(1.777 \pm 0.007)\si{\second}$ und $T_{2,r} = (1.779 \pm 0.038)\si{\second}$. Da die aufgezeichneten Längen ($ca 52cm$) nicht zu den gemessenen Zeiten passen, wird im weiteren mit $l_2 = g \frac{T_{2,l}^2}{4\pi ^2} = (75.8 \pm 2.0) \si{\centi \meter}$ gerechnet.

\paragraph{Theoriewerte}
Da sich lediglich die Eigenschwingungen der einzelnen Pendel als Theoriewerte berechnen lassen, seien sie hier aufgelistet:
\begin{align*}
  T_{1,l,t} &= (1.5308 \pm 0.0006) \si{\second}   \\
  T_{1,r,t} &=  (1.5335 \pm 0.0006) \si{\second}  \\
  T_{2,t} &= (1.747 \pm 0.023) \si{\second}
\end{align*}
Auf eine Auflistung der der entsprechenden $\omega$ wird an dieser Stelle verzichtet, da sie unmittelbar mit den Umlaufzeiten zusammenhängen.

\paragraph{gemessene Werte}
Im ersten Durchlauf ergibt sich die Schwingungsdauer der gleichsinnigen Schwingung zu $T_{1,+} = (1.446 \pm 0.035)\si{\second}$, die der gegensinnigen zu $T_{1,-} = (1.284 \pm 0.037)\si{\second}$. Dies entspricht den Frequenzen $\omega_{1,+} = (4.35 \pm 0.11)\si{\second}$ und $\omega_{1,-} = (4.89 \pm 0.14)\si{\second}$.
Im zweiten Durchlauf ergibt sich die Schwingungsdauer der gleichsinnigen Schwingung zu $T_{2,+} = (1.746 \pm 0.023)\si{\second}$, die der gegensinnigen zu $T_{1,-} = (1.559 \pm 0.030)\si{\second}$. Dies entspricht den Frequenzen $\omega_{1,+} = (3.60 \pm 0.05)\si{\second}$ und $\omega_{1,-} = (4.03 \pm 0.08)\si{\second}$.

Für die einzelnen Schwebungen werden gemessen:
\begin{align*}
  T_{1,s} &= (9.828 \pm 0.226) \si{\second} \\
  T_{2,s} &= (15.571 \pm 0.618) \si{\second}
\end{align*}

Bei der Messung für fünf Schwebungen ergeben sich:
\begin{align*}
  T_{1,5s} &= (10.2536 \pm 0.2632) \si{\second}\\
  T_{2,5s} &= (15.7474 \pm 0.1876) \si{\second}
\end{align*}

Die Kopplungskonstanten $K$ und $\kappa$ errechnen sich durch \eqref{eqn:ggs} und \eqref{eqn:K} zu:
\begin{align*}
  K_{1} &= (2.10 \pm 0.04) \si{\meter\per\second\squared} \\
  \kappa_{1} &= 0.12 \pm 0.04\\
  K_{2} &= (-0.62 \pm 0.17) \si{\meter\per\second\squared}\\
  \kappa_{2} &= 0.113 \pm 0.023
\end{align*}
