\section{Auswertung}
\label{sec:Auswertung}

\paragraph{Verwendete Pendel}
Im ersten Durchlauf werden die Pendel auf $l_{1,l} =(58.23 \pm 0.03)\si{\centi\meter}$ und $l_{1,r} =(58.43 \pm 0.03)\si{\centi\meter}$ eingestellt, sodass sich Schwingungsdauern von $T_{1,l} =(1.479 \pm 0.008)\si{\second}$ und $T_{1,r} = (1.466 \pm 0.006)\si{\second}$ einstellen.
Beim zweiten Durchlauf ergeben sich $T_{2,l}=(1.777 \pm 0.003)\si{\second}$ und $T_{2,r} = (1.779 \pm 0.016)\si{\second}$. Da die aufgezeichneten Längen ($ca 52\si{\centi \meter}$) nicht zu den gemessenen Zeiten passen, wird im weiteren mit $l_2 = g \frac{T_{2,l}^2}{4\pi ^2} = (75.8 \pm 2.0) \si{\centi \meter}$ gerechnet.

\paragraph{Theoriewerte}
Da sich lediglich die Eigenschwingungen der einzelnen Pendel als Theoriewerte berechnen lassen, seien sie hier aufgelistet:
\begin{align*}
  T_{1,l,t} &= (1.5308 \pm 0.0004) \si{\second}   \\
  T_{1,r,t} &=  (1.5335 \pm 0.0004) \si{\second}  \\
  T_{2,t} &= (1.747 \pm 0.023) \si{\second}
\end{align*}
Auf eine Auflistung der der entsprechenden $\omega$ wird an dieser Stelle verzichtet, da sie unmittelbar mit den Umlaufzeiten zusammenhängen.

\begin{table}
  \caption{Längen der Pendel}
  \label{tab:l}
  \sisetup{round-mode = places , round-precision = 3}
  \begin{tabular}{S S S S}
    \toprule
    {$l_{1,l}/\si{\centi\meter}$} & {$l_{1,r}/\si{\centi \meter}$} & {$l_{2,l} /\si{\centi \meter}$} & {$\l_{2,r}/\si{\centi \meter}$}\\
    \midrule
    58.2                          &  58.5                         & 52.7                           & 52.5 \\
    58.3                          &  58.4                         & 52.8                           & 52.6 \\
    58.2                          &  58.4                         & 52.8                           & 52.4 \\
    \bottomrule
  \end{tabular}
\end{table}

\begin{table}
  \caption{Schwingungsdauern der gekoppelten Pendel im ersten Durchlauf}
  \label{tab:T1}
  \sisetup{round-mode = places , round-precision = 3}
  \begin{tabular}{S S S S}
    \toprule
    {$T_{1,+}/\si{\second}$} & {$T_{1,-}/\si{\second}$} & {$T_{1,s}/\si{\second}$} & {$T_{1,5s}/\si{\second}$} \\
    \midrule
    7.6   & 6.32 & 9.87   & 10.14 \\
    7.06  & 6.27 & 9.86   & 10.14 \\
    7.03  & 6.6  & 9.9    & 10.78 \\
    6.95  & 6.73 & 9.9    & 10.17 \\
    7.24  & 6.72 & 10.16  & 10.07 \\
    7.21  & 6.24 & 9.4    & 10.08 \\
    7.34  & 6.29 & 9.41   & 10.18 \\
    7.35  & 6.23 & 9.92   & 10.18 \\
    7.23  & 6.44 & 9.96   & 10.10 \\
    7.27  & 6.36 & 9.9    & 10.10 \\
    \bottomrule
  \end{tabular}
  \end{table}

  \begin{table}
    \caption{Schwingungsdauern der gekoppelten Pendel im zweiten Durchlauf}
    \label{tab:T2}
    \sisetup{round-mode = places , round-precision = 2}
    \begin{tabular}{S S S S}
      \toprule
      {$T_{2,+}/\si{\second}$} & {$T_{2,-}/\si{\second}$} & {$T_{2,s}/\si{\second}$} & {$T_{2,5s}/\si{\second}$} \\
      8.7   & 7.75 & 16.64 & 15.69 \\
      8.7   & 7.73 & 16.69 & 15.69 \\
      8.73  & 7.53 & 14.73 & 15.73 \\
      8.7   & 7.58 & 14.78 & 15.73 \\
      8.52  & 7.76 & 15.53 & 15.72 \\
      8.56  & 7.82 & 15.55 & 15.73 \\
      8.86  & 7.92 & 15.46 & 16.09 \\
      8.86  & 7.92 & 15.50 & 16.09 \\
      8.81  & 7.92 & 15.53 & 15.51 \\
      8.86  & 8.03 & 15.30 & 15.52 \\
      /     & /    & 15.32 & / \\
      /     & /    & 15.5  & / \\
  \bottomrule
\end{tabular}
\end{table}
\FloatBarrier
\paragraph{gemessene Werte}
Im ersten Durchlauf ergibt sich die Schwingungsdauer der gleichsinnigen Schwingung zu $T_{1,+} = (1.446 \pm 0.011)\si{\second}$, die der gegensinnigen zu $T_{1,-} = (1.284 \pm 0.012)\si{\second}$. Dies entspricht den Frequenzen $\omega_{1,+} = (4.35 \pm 0.03)\si{\hertz}$ und $\omega_{1,-} = (4.89 \pm 0.04)\si{\hertz}$.
Im zweiten Durchlauf ergibt sich die Schwingungsdauer der gleichsinnigen Schwingung zu $T_{2,+} = (1.746 \pm 0.007)\si{\second}$, die der gegensinnigen zu $T_{2,-} = (1.559 \pm 0.009)\si{\second}$. Dies entspricht den Frequenzen $\omega_{2,+} = (3.60 \pm 0.02)\si{\hertz}$ und $\omega_{2,-} = (4.03 \pm 0.03)\si{\hertz}$.
Es ergeben sich mit diesen Werten für die Schwebungsfrequenz gemäß \eqref{eqn:gks} die Erwartungswerte $\omega{1,s,e} = (0.55 \pm 0.06) \si{\hertz}$ und $\omega_{2,s,e}(0.43 \pm 0.03) \si{\hertz}$.
Für die einzelnen Schwebungen werden gemessen:
\begin{align*}
  T_{1,s} &= (9.828 \pm 0.071) \si{\second} \\
  T_{2,s} &= (15.571 \pm 0.195) \si{\second}
\end{align*}
Dies entspricht
\begin{align*}
  \omega_{1,s} &= (0.64 \pm 0.01) \si{\hertz} \\
  \omega_{2,s} &= (0.40 \pm 0.01) \si{\hertz}
\end{align*}
Bei der Messung für fünf Schwebungen ergeben sich:
\begin{align*}
  T_{1,5s} &= (10.2536 \pm 0.0832) \si{\second}\\
  T_{2,5s} &= (15.7474 \pm 0.0593) \si{\second}
\end{align*}
Dies entpsricht
\begin{align*}
  \omega_{1,5s} &= (0.61 \pm 0.01) \si{\hertz}\\
  \omega_{2,5s} &= (0.40 \pm 0.02) \si{\hertz}
\end{align*}

Die Kopplungskonstanten $K$ und $\kappa$ errechnen sich durch \eqref{eqn:ggs} und \eqref{eqn:K} zu:
\begin{align*}
  K_{1} &= (2.08 \pm 0.13) \si{\meter\per\second\squared} \\
  \kappa_{1} &= 0.12 \pm 0.01\\
  K_{2} &= (1.25 \pm 0.18) \si{\meter\per\second\squared}\\
  \kappa_{2} &= 0.113 \pm 0.007
\end{align*}
Setzt man diese Werte für $\kappa$ erneut in die Formeln \eqref{eqn:K} ein, so erhält man die Werte $\omega_{1,-} = (4.89 \pm 0.05) \si{\hertz}$ und $\omega_{2,-} = (4.03 \pm 0.02)\si{\hertz}$. Hiermit wiederum ergeben sich $\omega_{1,s,t} = (0.79 \pm 0.05) \si{\hertz}$ und $\omega_{2,s,t} = (0.43 \pm 0.05) \si{\hertz}$.
