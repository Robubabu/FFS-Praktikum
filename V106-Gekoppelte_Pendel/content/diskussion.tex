\section{Diskussion}
\label{sec:Diskussion}
Im ersten Durchlauf stimmen die gemessenen Schwingungsdauern für die gleichphasige Schwingung bis auf $\Delta T_{1,l} = (0.0518 \pm 0.0206) \si{\second}$, bzw. $\Delta T_{1,r} = (0.0675 \pm 0.0146) \si{\second}$ mit den Theoriewerten überein. Dies stellt eine sehr gute Übereinstimmung dar, angesichts der Tatsache, dass alle Messungen von Hand und per Augenmaß erfolgten. Für den zweiten Durchlauf liegt die Abweichung in der gleichen Größenordnung, was angesichts der Tatsache, dass die Länge aus eben jenen Zeiten berechnet wurde, keine Aussagen zulässt.
Bei den Schwebungen erhält man zwischen den einzeln und gruppenweise Abweichungen von $\Delta T_{1,s} = (0.4256 \pm 0.4892)\si{\second}$, bzw. $\Delta T_{1,s} = (0.1764 \pm 0.8056)\si{\second}$, welche, da die Fehler jeweils größer als die eigentliche Abweichung sind, als vernachlässigbar klein angesehen werden. Allerdings fällt die gruppenweise Messung im zweiten Durchgang etwas genauer aus (Fehler $<1/3$ von dem bei der Einzelmessung). Dies ist darin begründet, dass es schwer ist, Schwebungen optisch exakt zu erfassen, und bei der Gruppenmessung der dadurch entstehende Fehler auf fünf Schwebungen verteilt, dh. gefünftelt, wird.
