\section{Diskussion}
\label{sec:Diskussion}
Im ersten Durchlauf stimmen die gemessenen Schwingungsdauern für die gleichphasige Schwingung bis auf $\Delta T_{1,l} = (0.0518 \pm 0.0084) \si{\second}$, bzw. $\Delta T_{1,r} = (0.0675 \pm 0.0064) \si{\second}$ mit den Theoriewerten überein. Dies stellt eine sehr gute Übereinstimmung dar, angesichts der Tatsache, dass alle Messungen von Hand und per Augenmaß erfolgten. Für den zweiten Durchlauf liegt die Abweichung in der gleichen Größenordnung, was angesichts der Tatsache, dass die Länge aus eben jenen Zeiten berechnet wurde, keine Aussagen zulässt.
Bei den Schwebungen erhält man zwischen den einzeln und gruppenweise Abweichungen von $\Delta T_{1,s} = (0.4256 \pm 0.1542)\si{\second}$, bzw. $\Delta T_{2,s} = (0.1764 \pm 0.2543)\si{\second}$. Für den zweiten Durchlauf wird diese, da der Fehler größer als die eigentliche Abweichung ist, als vernachlässigbar klein angesehen. Auch beim ersten Durchlauf ist der Fehler dieser Differenz mit etwa $36\%$ sehr groß. Allerdings fällt die gruppenweise Messung im zweiten Durchgang etwas genauer aus (Fehler $<1/3$ von dem bei der Einzelmessung). Dies ist darin begründet, dass es schwer ist, Schwebungen optisch exakt zu erfassen, und bei der Gruppenmessung der dadurch entstehende Fehler auf fünf Schwebungen verteilt, dh. gefünftelt, wird.
Bei der Schwebung fällt jedoch auf, dass bei der ersten Messreihe enorme Differenzen zwischen $\omega_{1,s}$, $\omega_{1,5s}$ auf der einen Seite und $\omega_{1,s,t}$ ($\Delta \omega \approx (0.15 \pm 0.06) \si{\hertz}$ vorliegen. Diese Tatsache ist darin zu begründen, dass der "Theoriewert" für $\omega_s$ über mehrere Schritte aus den Messwerten selbst berechnet wurde und somit von massiver Fehlerfortpflanzung betroffen ist.
