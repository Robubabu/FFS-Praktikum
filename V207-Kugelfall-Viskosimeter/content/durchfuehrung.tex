\section{Durchführung}
\label{sec:Durchführung}
Zunächst ist es nötig, das Viskosimeter (Abb. \ref{fig:Viskosimeter}) mit destilliertem Wasser zu befüllen und mit Hilfe der Libelle so aus zu richten, dass die Rohrhalterung lotrecht steht. Nun werden für zwei unterschiedlich große Glaskugeln Durchmesser und Gewicht bestimmt und für jede Kugel zehn Fallzeiten gemessen. Mit bekannter Apparaturkonstante $K_{kl}$ kann aus \eqref{eqn:eta} $K_{gr}$ für die größere der beiden Kugeln bestimmt werden. Besonders wichtig ist hierbei das Vermeiden von Bläschen im Messrohr.
Anschließend wird das Viskosimeter über das umliegende Wasserbad schrittweise auf $70 \si{\celsius}$ erwärmt (in 10 Schritten) und zwischen jedem Heizschritt wird die Fallzeit der großen Kugel zweimal gemessen. Hierbei ist der bei der Erwärmung entstehende Druck über das Druckventil des Viskosimeters stets auszugleichen.
