\section{Diskussion}
\label{sec:Diskussion}
\subsection{Mathematische Methoden}
Alle im Folgenden brechneten relativen Fehler werden mit der Formel
\begin{equation*}
  \bar{x} = \frac{\lvert x_m - x_e \rvert}{\lvert x_e \rvert} \cdot 100 \%
\end{equation*}
brechnet. Dabei bezeichnet $\bar{x}$ den relativen Fehler, $ x_m $ den Messwert
und $x_e$ den Erwartungs-bzw. Literaturwert.
\subsection{Zur Bestimmung der Apparatekonstante}
Da die Appartekonstante vom Versuchsapparat abhängig ist, ist es schwer
aussagekräftige Literaturwerte zu finden. Aber da diese Konstante abhängig
von der gemessenen Viskosität ist, wird diese folglich mit einem Literaturwert
\cite{wiki2} verglichen. Dann ergibt sich für die gemessene Viskosität bei
 20 \si{\celsius} ein relativer Fehler von $(20.84 \pm 0.32)\si{\percent}$.
Da die Temperatur nicht innerhalb des Fallrohrs gemessen wurde kann es durchaus
sein, dass es durch eine falsch bestimmte Temperatur zur dieser Abweichung
gekommen ist. Eine solche Abweichung kann nur durch Messfehlern in der Thermik
entstanden sein, d.h. Temperatur-/Druckunterschiede.
\subsection{Zur Temperaturabhängigkeit und Viskosität}
Hier kann kein aussagekräftiger relativer Fehler bestimmt werden. Trotdem
ist aus den Ergebnissen ersichlich, dass die Viskosität abnimmt für eine steigende
Temperatur, was auch aus den Literaturwerten hervor geht.
\subsection{Zur Reynolds-Zahl}
Die brechneten Reynolds-Zahlen zeigen eindeutig, dass eine laminare Strömung
vorliegt und somit kein systematischer Fehler vorliegt. 
