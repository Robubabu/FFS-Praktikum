\section{Diskussion}
\label{sec:Diskussion}

\paragraph{Einseitige Einspannung}

Da die Dichten beider Stäbe $\left(\rho_r =8397 \frac{kg}{m^3}, \rho_q = 8373 \frac{kg}{m^3}\right)$ nur um $24 \frac{kg}{m^3}$, also ca. $0.29 \%$ voneinander abweichen, kann ohne weitere Bedenken von identischen Materialien ausgegangen werden. Dementsprechend sollten die Elastizitätsmodule ebenfalls identisch sein. tatsächlich ergibt sich zwischen dem Elastizitätsmodul des runden und des eckigen Stabes eine Differenz von $\Delta E = \left(7.93 \pm 0.53 \right) \cdot 10^{9} \frac{N}{m^2} \hat{\approx} (7.3 \pm 0.5) \% $ was eine minimale Abweichung der beiden Elastizitätsmodule darstellt. Dies stellt ein erstaunlich exaktes Ergebnis da angesichts des Zustandes des Messaufbaus. Die fehlende Mutter an der Halterung der Uhren an der Messschiene sorgte für eine Toleranz der Auslenkung von ca $\pm 3mm$, welche durch vorsichtige Handhabung nur bedingt ausgeglichen werden kann.

\paragraph{Beidseitige Auflage}

In dieser Messreihe fällt vor allem der Unterschied in den Auslenkungen der beiden Seiten auf ($0.655mm$ zwischen $x=25cm$ und $x = 30cm$), auf. Dies ist eine Folge der bereits erwähnten Schwächen der Apparatur.
Die Berrechnung des Elastizitätsmoduls liefert allerdings ein relativ präzises Ergebniss (ca. $4 \%$ Abweichung), was jedoch deutlich ungenauer als die einseitig eingespannte Messung ist. Dies ist vor allem in der geringen Auslenkung des Stabes sowie der wesentlich geringeren Anzahl an Messwerten begründet.
