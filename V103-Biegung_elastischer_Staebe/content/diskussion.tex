\section{Diskussion}
\label{sec:Diskussion}

\subsection{Einseitige Einspannung}
Da die Dichten beider Stäbe $\left(\rho_r =8397 \frac{kg}{m^3}, \rho_q = 8373 \frac{kg}{m^3}\right)$ nur um $24 \frac{kg}{m^3}$, also ca. $0.29 \%$ voneinander abweichen, kann ohne weitere Bedenken von identischen Materialien ausgegangen werden. Dementsprechend sollten die Elastizitätsmodule ebenfalls identisch sein. tatsächlich ergibt sich zwischen dem Elastizitätsmodul des runden und des eckigen Stabes eine Differenz von $\Delta E = \left(3.74e9 \pm 10.4e9 \right) \frac{N}{m^2} \hat{\approx} (38 \pm 77) \% $ was eine starke Abweichung der beiden Elastizitätsmodule darstellt. Allerdings sind bereits die Messunsicherheiten von $E$ ($44 \%$ bzw. $45$) enorm, was vor allem an den nicht ordnungsgemäß befestigten und somit wackelnden Messuhren liegen dürfte. Die fehlende Mutter an der Halterung der Uhren an der Messschiene sorgte für eine Toleranz der Auslenkung von ca $\pm 3mm$, welche durch vorsichtige Handhabung nur bedingt ausgeglichen werden kann.

\subsection{Beidseitige Auflage}

In dieser Messreihe fällt vor allem der Unterschied in den Auslenkungen der beiden Seiten auf ($0.655mm$ zwischen $x=25cm$ und $x = 30cm$), welche bei der Bestimmung des Elastizitätsmoduls des runden Stabes für die linke und rechte Seite zu eine Differenz von $\Delta E \approx \left(51.7e8 \pm 9.5e8\right) \frac{N}{m^2}$. Die Messung dieser Werte liefert allerdings erstaunlich präzise Ergebnisse, betrachtet man die Seiten einzeln (links ca $7.5 \%$ Abweichung, rechts sogar nur $0.38 \%$), was vermuten lässt, dass diese Art der Messung deutlich genauer ist, jedoch ein systematischer Fehler in Form einer festen Differenz zwischen beiden Messuhren vorliegt.
