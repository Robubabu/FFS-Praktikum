\section{Diskussion}
\label{sec:Diskussion}
Wackelkontakt dies das.
Temperaturungenauigkeit kochendes Wasser, schwör.

\paragraph{Blei}

Bei den zum Vergleich stehenden Werten $c_{pb}$ und $c_{pb,t}$ fällt sofort die starke Abweichung von $\Delta c \approx 41 \%$ auf, welche Zweifel an der Gültigkeit des Dulong-Petitschen Gesetzes begründet. Allerdings ist zu beachten, dass zur Temperaturbestimmung die Temperatur des kochenden Wassers mit exakt $100 \si{\celsius}$ angenommen wurde. Da der Siedepunkt von Wasser jedoch druckabhängig ist, können hier Ungenauigkeiten herrühren, welche sich über die daraus resultierenden Abweichungen von $a$ fortpfanzen können. Weiterhin ist zu bedenken, dass das isolierende Gefäß, indem der Probekörper mit dem Wasser gekühlt wurde, nicht perfekt isolieren kann und zudem über keinen Deckel verfügte, weshalb die Wärmemenge nicht konstant im System erhalten bleibt. Zuletzt bleiben noch die starken Abweichungen, welche durch Wackelkontakte der zur Verfügung stehenden Thermoelemente entstehen.
Zieht man alle diese Aspekte in Betracht, so lässt sich dennoch sagen, dass das Dulong-Petitsche Gesetz für Blei doch eine recht Gute Näherung für die spezifische Wärmekapazität liefert.

\paragraph{Graphit}
Die Abweichungen des gemessenen $c_{gr}$ zum nach Dulong-Petitschem Gesetz erwarteten $c_{gr,t}$ sind mit $\Delta c \approx 214 \%$ so enorm, dass klar ersichtlich ist, dass für Graphit Quantendynamische Vorgänge nicht vernachtlässigt werden können.

\paragraph{Kupfer}
Die Kupfermessung liefert $\Delta c \approx 50 \%$. Auch hier liefert das Dulong-Petitsche Gesetz also hinreichend gute Werte.

\paragraph{Fazit}
Aufgrund des recht simplen Versuchsaufbaus lassen sich aus den Ergebnissen dieses Versuches kaum Erkenntnisse ziehen, doch legen die ausgewerteten Daten dennoch nahe, dass das Dulong-Petitsche Gesetz für leichte Stoffe kaum Gültigkeit besitzt, für hinreichend SChwere Stoffe jedoch durchaus als zulässige Näherung betrachtet werden kann.
