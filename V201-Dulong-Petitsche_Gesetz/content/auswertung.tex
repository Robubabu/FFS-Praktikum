\section{Auswertung}
\label{sec:Auswertung}
\paragraph{Eichung}
Für den Zusammmenhang zwischen am Thermoelement auftretender Spannung und Temperatur erhält man:

\begin{align*}
  a &= \left(61.65 \pm 0.34 \right) \si{\kelvin \per \milli \volt} \\
  b &= \left(332.03 \pm 0.33 \right) \si{\kelvin}.
\end{align*}

\paragraph{Bestimmung der Wärmekapazität des Isolierenden Gefäßes}
Für die $0.3 \si{\liter}$ heißes Wasser werden $(0.510 \pm 0.001) \si{\milli \volt}$ am Thermoelement gemessen, dies entspricht einer Temperatur von $(363.5 \pm 0.5) \si{\kelvin}$. Für das kalte Wasser ergibt sich $-(0.557 \pm 0.001)\si{\milli \volt}$ ($(297.7 \pm 0.2) \si{\kelvin}$) und für das gemischte Wasser  $-(0.103 \pm 0.001)\si{\milli \volt}$ ($(325.7 \pm 0.3) \si{\kelvin}$). Die Wärmekapazität des Gefäßes beträgt somit $m_gc_g = (441 \pm 8) \si{\joule \per \kelvin}$.

\paragraph{Wärmekapazität von Blei}

\begin{table}
  \centering
  \caption{gemessene Thermoelementspannungen (Abweichungen $\pm 0.001 \si{\milli \volt}$) bei Blei.}
  \label{tab:Ublei}
  \begin{tabular}{S S S}
    \toprule
    {$U_{pb} / \si{\milli \volt}$} & {$U_w / \si{\milli \volt}$} & {$U_m / \si{\milli \volt}$} \\
    \midrule
    0.630 & -0.148 & -0.137 \\
    0.650 & -0.136 & -0.110 \\
    0.652 & -0.132 & -0.084 \\
    \bottomrule
  \end{tabular}
\end{table}


\paragraph{Wärmekapazität von Graphit}


\begin{table}
  \centering
  \caption{gemessene Thermoelementspannungen (Abweichungen $\pm 0.001 \si{\milli \volt}$) bei Graphit.}
  \label{tab:Ugraphit}
  \begin{tabular}{S S S}
    \toprule
    {$U_{gr} / \si{\milli \volt}$} & {$U_w / \si{\milli \volt}$} & {$U_m / \si{\milli \volt}$} \\
    \midrule
    0.640 & -0.189 & -0.177 \\
    0.660 & -0.150 & -0.120 \\
    0.670 & -0.135 & -0.120 \\
    \bottomrule
  \end{tabular}
\end{table}

\paragraph{Wärmekapazität von Kupfer}

\begin{table}
  \centering
  \caption{gemessene Thermoelementspannungen (Abweichungen $\pm 0.001 \si{\milli \volt}$) bei Kupfer.}
  \label{tab:Ukupfer}
  \begin{tabular}{S S S}
    \toprule
    {$U_{cu} / \si{\milli \volt}$} & {$U_w / \si{\milli \volt}$} & {$U_m / \si{\milli \volt}$} \\
    \midrule
    0.690 & -0.082 & -0.066 \\

    \bottomrule
  \end{tabular}
\end{table}
