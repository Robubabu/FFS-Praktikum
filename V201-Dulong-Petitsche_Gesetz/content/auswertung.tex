\section{Auswertung}
\label{sec:Auswertung}
\paragraph{Eichung}
Für den Zusammmenhang zwischen am Thermoelement auftretender Spannung und Temperatur erhält man:

\begin{align*}
  a &= \left(61.65 \pm 0.34 \right) \si{\kelvin \per \milli \volt} \\
  b &= \left(332.03 \pm 0.33 \right) \si{\kelvin}.
\end{align*}

\paragraph{Bestimmung der Wärmekapazität des Isolierenden Gefäßes}
Für die $0.3 \si{\liter}$ heißes Wasser werden $(0.510 \pm 0.001) \si{\milli \volt}$ am Thermoelement gemessen, dies entspricht einer Temperatur von $(363.5 \pm 0.5) \si{\kelvin}$. Für das kalte Wasser ergibt sich $-(0.557 \pm 0.001)\si{\milli \volt}$ ($(297.7 \pm 0.2) \si{\kelvin}$) und für das gemischte Wasser  $-(0.103 \pm 0.001)\si{\milli \volt}$ ($(325.7 \pm 0.3) \si{\kelvin}$). Die Wärmekapazität des Gefäßes beträgt somit $m_gc_g = (441 \pm 8) \si{\joule \per \kelvin}$.

\paragraph{Wärmekapazität von Blei}

\begin{table}
  \centering
  \caption{gemessene Thermoelementspannungen (Abweichungen $\pm 0.001 \si{\milli \volt}$) bei Blei, sowie die daraus errechneten Temperaturen und Wärmekapazitäten.}
  \label{tab:blei}
  \begin{tabular}{S S S S S S S}
    \toprule
    {$U_{pb} / \si{\milli \volt}$} & {$T_{pb} / \si{\kelvin}$} & {$U_w / \si{\milli \volt}$} & {$T_{w} / \si{\kelvin}$} & {$U_m / \si{\milli \volt}$} & {$T_{m} / \si{\kelvin}$} & {$c_{pb} / \si{\joule \per \kilo \gram \kelvin}$}\\
    \midrule
    0.630 & {$370.9 \pm 0.5$} & -0.148 &{$322.9 \pm 0.3$} & -0.137 & {$323.6 \pm 0.3$} & {$78 \pm 10$}\\
    0.650 & {$372.1 \pm 0.6$} &-0.136 & {$323.6 \pm 0.3$} &-0.110 & {$325.2 \pm 0.3$} & {$186 \pm 10$}\\
    0.652 & {$372.2 \pm 0.6$} &-0.132 & {$323.9 \pm 0.3$} &-0.084 & {$326.8 \pm 0.3$} & {$354 \pm 10$}\\
    \bottomrule
  \end{tabular}
\end{table}

Die für Blei bemessenen und berechneten Werte lassen sich Tab. \ref{tab:blei} entnehmen. Aus diesen Werten ergibt sich die gemittelte Wärmekapazität von Blei zu $c_{pb} = (206 \pm 3) \si{\joule \per \kilo \gram \kelvin}$. Die nach dem Dulong-Petitschen Gesetz erwartete Wärme Kapazität beträgt $c_{pb,t} = 3 R \approx 120 \si{\joule \per \kilo \gram \kelvin}$.


\paragraph{Wärmekapazität von Graphit}


\begin{table}
  \centering
  \caption{gemessene Thermoelementspannungen (Abweichungen $\pm 0.001 \si{\milli \volt}$) bei Graphit, sowie die daraus errechneten Temperaturen und Wärmekapazitäten.}
  \label{tab:graphit}
  \begin{tabular}{S S S S S S S}
    \toprule
    {$U_{gr} / \si{\milli \volt}$} & {$T_{gr} / \si{\kelvin}$} & {$U_w / \si{\milli \volt}$} & {$T_{w} / \si{\kelvin}$} & {$U_m / \si{\milli \volt}$} & {$T_{m} / \si{\kelvin}$} & {$c_{gr}/ \si{\joule \per \kilo \gram \kelvin}$}\\
    \midrule
    0.640 & {$371.5 \pm 0.5$} & -0.189 & {$320.4 \pm 0.3$} & -0.177 & {$321.1 \pm 0.3$} & {$403 \pm 48$}\\
    0.660 & {$372.7 \pm 0.6$} & -0.150 & {$322.8 \pm 0.3$} & -0.120 & {$324.6 \pm 0.3$} & {$1056 \pm 51$}\\
    0.670 & {$373.3 \pm 0.6$} & -0.135 & {$323.7 \pm 0.3$} & -0.120 & {$324.6 \pm 0.3$} & {$522 \pm 50$}\\
    \bottomrule
  \end{tabular}
\end{table}

Die für BGraphit gemessenen und berechneten Werte lassen sich Tab. \ref{tab:graphit} entnehmen. Aus diesen Werten ergibt sich die gemittelte Wärmekapazität von Graphit zu $c_{gr} = (660 \pm 28) \si{\joule \per \kilo \gram \kelvin}$. Die nach dem Dulong-Petitschen Gesetz erwartete Wärme Kapazität beträgt jedoch $c_{gr,t} = 3 R \approx 2073 \si{\joule \per \kilo \gram \kelvin}$.

\paragraph{Wärmekapazität von Kupfer}

\begin{table}
  \centering
  \caption{gemessene Thermoelementspannungen (Abweichungen $\pm 0.001 \si{\milli \volt}$) bei Kupfer, sowie die daraus errechneten Temperaturen und Wärmekapazitäten.}
  \label{tab:kupfer}
  \begin{tabular}{S S S S S S S}
    \toprule
    {$U_{cu} / \si{\milli \volt}$} & {$T_{cu} / \si{\kelvin}$} & {$U_w / \si{\milli \volt}$} & {$T_{w} / \si{\kelvin}$} & {$U_m / \si{\milli \volt}$} & {$T_{m} / \si{\kelvin}$} & {$c_{cu}/ \si{\joule \per \kilo \gram \kelvin}$}\\
    \midrule
    0.690 & {$374.6 \pm 0.6$} & -0.082 & {$327.0 \pm 0.3$} & -0.066 & {$328.0 \pm 0.3$} & {$263 \pm 23$}\\

    \bottomrule
  \end{tabular}
\end{table}

Die für Kupfer gemessenen und berechneten Werte lassen sich Tab. \ref{tab:Kupfer} entnehmen. Aus diesen Werten ergibt sich die Wärmekapazität von Kupfer zu $c_{cu} = (263 \pm 23) \si{\joule \per \kilo \gram \kelvin}$. Die nach dem Dulong-Petitschen Gesetz erwartete Wärme Kapazität beträgt $c_{cu,t} = 3 R \approx 130 \si{\joule \per \kilo \gram \kelvin}$.
