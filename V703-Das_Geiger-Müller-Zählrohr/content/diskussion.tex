\section{Diskussion}
\label{sec:Diskussion}
Die Charakteristik ist einzigartig für das jeweilige Zählrohr weshalbt nur
schwer Literaturwerte gefunden werden können.
Die freigetzte Ladungsmenge ist so gesehen von der anzulegende Spannung
abhängig also auch abhängig von der Bauweise des Zählrohrs. Es wurden also
keine Literaturwerte gefunden. Da die Werte von der oszillographischen
Bestimmung und der Zwei-Quellen-Methode von zwei verschiedenen Appparturen
kommen können diese auch nicht verglichen werden. Die Steigung der Charakteristik
ist gering sollte aber für ein ideales Zählrohr gegen Null tendieren.


% **ALT**
% \subsection{Zur freigesetzten Ladungsmenge und Charakteristik des Zählrohrs}
% Die Charakteristik ist einzigartig für das jeweilige Zählrohr weshalbt nur
% schwer Literaturwerte gefunden werden können.
% Die freigetzte Ladungsmenge ist so gesehen von der anzulegende Spannung
% abhängig also auch abhängig von der Bauweise des Zählrohrs. Es wurden also
% keine Literaturwerte gefunden.
%
% \subsection{Vergleich der gemessenen Totzeiten}
% Zum Vergleich der gemessenen Werte wird der relative Fehler wie folgt bestimmt:
% \begin{equation*}
%   \bar{t_O} = \frac{\lvert t_{ZQM} - t_O \rvert}{\lvert t_{ZQM} \rvert}\;.
% \end{equation*}
% Dabei bezeichnet $t_O$ den Wert der Totzeit die mit dem Oszilloskop und
% $t_{ZQM}$ die mit dem Zwei-Quellen-Methode bestimmt wurde.
% So ergibt sich das die Werte um
% \begin{equation*}
%   \SI{8(2)e3}{\percent}
% \end{equation*}
% abweichen. Da so eine Abweichung nicht durch Messungenauigkeiten erklärt werden
% kann ist festzuhalten, dass die oszillographische Messung der Todzeit sehr
% ungenau ist.
