\section{Durchführung}
\label{sec:Durchführung}

In diesem Teil wird das Zählrohr von einer $\beta$-Strahlenquelle bombardiert und die Spannung varriiert, um die Abhängigkeit der Zählrate von der Spannung zu untersuchen. Außerdem wird versucht, die Nachentladungen an einem Oszilloskop sichtbar zu machen. Mit dem Oszilloskop wird auch die Totzeit abgeschätzt, indem grob abgelesen wird, wie lang die Abklingzeit der Impulse ist.

Um die Totzeit genauer zu bestimmen wird  eine zweite Quelle hinzugezogen und nach der Zweiquellenmethode gemäß \eqref{eqn:2Quellen}, bzw \eqref{eqn:easy} vorgegangen.
