\section{Diskussion}
\label{sec:Diskussion}
\subsection{Zu 3.1}
  Es ist ersichtlich, dass wir für den Fit der unteren Einhüllenden, Werte aus gelassen haben, da
  diese keine Aussage über den Abfall der Schwingung machen, sonder Ergebniss des Anschwingvorganges sind.Diese
  hätten das Ergebnis sonst verfälscht. Die Werte entsprechen den Erwartungswerten, auch wenn es erst
  nicht so scheint denn man muss noch den Widerstand des RC-Generators mit ein beziehen der bei ca. 50 $\Omega$ liegt.
  \subsection{Zu 3.2}
  Die Abweichung der Widerstände lässt sich durch Versuchsapperatur interne Wiederstände und Kopplungen erklären
  und liegen nach unserem Ermessen noch innerhalb der Toleranz.
  \subsection{Zu 3.3}
  Dort muss uns ein Fehler mit der Größenordnung der Konstantenspannung U unterlaufen sein den außer
  der Dezimalstellen sind die Werte entsprechend.
  \subsection{Zu 3.4}
    Aufgrund zu weniger Werte lassen sich hier keine weiteren Erkenntnisse rausziehen.
  \subsection{Zu 3.5}
   Das ansteigen des Scheinwiederstandes muss durch die Versuchsapperatur hervorgerufen werden, wie und warum dies geschieht ist uns aber unbekannt.
