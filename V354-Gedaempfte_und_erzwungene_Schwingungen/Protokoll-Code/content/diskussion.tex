\section{Diskussion}
\label{sec:Diskussion}
\subsection{Zur Bestimmung des effektiven Dämpfungswiderstandes für den Fall
der gedämpften Schingung}
  Es ist ersichtlich, dass für den Fit der unteren Einhüllenden,
  Werte aus gelassen wurden, da
  diese keine Aussage über den Abfall der Schwingung machen,
   sonder Ergebniss des Anschwingvorganges sind (s.Abb.\ref{fig:5aplot}).
  Die relativen Fehler der effektiven Widerstände der jeweiligen Einhüllenden
  sind
  \begin{equation*}
    \increment R_{oben} = (13,3 \pm 2,2) \% \quad \text{und für}\quad
    \increment R_{unten} = (11 \pm 4) \%.
  \end{equation*}
Daraus ist zu schließen, dass Messwerte und Fit von relativ guter Qualtität sind.
  \subsection{Zur groben Messung der Dämpfungswiderstandes}
  Die Abweichung der Widerstände lässt sich durch Versuchsapperatur interne Wiederstände und Kopplungen erklären
  und liegen noch innerhalb der Toleranz.
  \subsection{Zur Bestimmung der Frequenzabhängigkeit der Kondensatorspannung }
  Dort muss uns ein Fehler mit der Größenordnung der Konstantenspannung U unterlaufen sein den außer
  der Dezimalstellen sind die Werte entsprechend.
  \subsection{Zur Bestimmug der Frequentabhängigkeit der Phase zwischen
  Erreger-und Kondensatorspannung}
    Aufgrund der Tatsache, dass im interessanten Bereich
    vor dem Resonanzpeak keine Werte vorhanden sind,
    lassen sich hier keine weiteren Erkenntnisse rausziehen.
  \subsection{Zur Bestimmung der Frequenzahängigkeit der Impedanz}
   Das ansteigen des Scheinwiederstandes im Bereich von ca. 35 bis
   bis 50 kHz (s. Abb. \ref{fig:5eplot}) muss durch die Versuchsapperatur
   hervorgerufen werden.
