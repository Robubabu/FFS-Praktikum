\section{Diskussion}
\label{sec:Diskussion}
\subsection{Methoden.}
Alle relativen Fehler wurden nach der Formel
\begin{equation*}
  \tilde{x} = \frac{ \lvert x_{lit} - x_{mess} \rvert}{\lvert x_{lit} \rvert}
  \cdot 100 \%
\end{equation*}
berechnet, dabei bezeichnet $x_{lit}$ den Literaturwert der Messgröße $x_{mess}$.

\subsection{Zur Bestimmung der Brennweite über die Gegenstands- und Bildweite.}
Zur quantitativen Diskussion der Bestimmung der Brennweite anhand der Gleichung
\eqref{eqn:LG} wurde der relative Fehler der gemittelten Brennweite
\SI{0.0983(4)}{\meter} zur Herstellerangabe von \SI{0.1}{\meter} Brennweite
bestimmt, dieser beträgt \SI{1.7(4)}{\percent}. Ebenfalls wurde der relative
Fehler des Abbildungsmaßstabes bestimmt über die Gegenstands- und Bildweiten
\SI{1.5(6)}{\meter}
zum Abbildungsmaßstabes bestimmt über die Gegenstands- und Bildgröße
\SI{1.4(6)}{\meter} bestimmt, dieser beträgt \SI{0(6)e1}{\percent}.
Im Rahmen der vom Versuchsaufbau gegeben Genauigkeit können die Gleichungen
\eqref{eqn:AG} und \eqref{eqn:LG} als verifiziert angesehen werden.

\subsection{Zur Methode nach Bessel.}
Es wurde zur quantitativen Diskussion der relative Fehler von
der mittleren Brennweite zur Herstellerangabe von \SI{0.1}{\meter} bestimmt.
Für normales Licht wurde eine Brennweite von \SI{0.040(6)}{\meter} bestimmt,
der relative Fehler beträgt also \SI{60(6)}{\percent}.
Für rotes und blaues Licht wurde eine Brennweite von \SI{0.034(6)}{\meter}
bestimmt, der relative Fehler beträgt \SI{66(6)}{\percent}. Da die Abweichungen
so groß sind können die Abweichungen nur dadurch erklärt werden, dass eine
Linse mit einer anderen Brennweite verwendet wurde.

\subsection{Zur Methode nach Abbe.}
Um die Werte mit der Theorie vergleichen zu können, wird der Mittelwert
von $f_{b'}$ und $f_{g'}$ bestimmt und mit dem Ergebnis der Gleichung
\begin{equation*}
  \frac{1}{f_{System}} = \frac{1}{f_{1}} + \frac{1}{f_{2}} + \frac{d}{f_{1} f_{2}}
\end{equation*}
verglichen. Dabei bezeichnen $f_1$ und $f_2$ die vom Hersteller gegebenen
Brennweiten und $d$ den Abstand der Mittelpunkte der Linsen. Eigentlich
sollten sich die Brechkräfte addieren (s. Gleichung \eqref{eqn:D}) aber
da dieser Zusammenhang nur für dünne Linsen gilt muss noch ein Korrekturterm
angefügt werden.
Für $f_1 = \SI{100}{\milli\meter}$, $f_2 = \SI{-100}{\milli\meter}$ und
$d = \SI{6}{\centi\meter}$ ergibt sich für die Brennweite des
Systems $f_{System} =\SI{167}{\milli\meter}$. Nun wird der relative
Fehler von dem Mittelwert der Brennweiten $f_{b'}$ und $f_{g'}$ zu der
neue berechneten Brennweite des System bestimmt, der relative Fehler beträgt
\SI{23(6)}{\percent}. Der Mittelwert der Brennweiten $f_{b'}$ und $f_{g'}$
beträgt \SI{129(10)}{\milli\meter}. Die Abweichung kann durch die sphärische
Abberation und das erschwärte Bestimmen eines scharfen Bildes durch Lichtbrechung
erklärt werden.
