\section{Diskussion}
\label{sec:Diskussion}
\subsection{Methoden.}
Alle relativen Fehler wurden nach der Formel
\begin{equation*}
  \tilde{x} = \frac{ \lvert x_{lit} - x_{mess} \rvert}{\lvert x_{lit} \rvert}
  \cdot 100 \%
\end{equation*}
berechnet, dabei bezeichnet $x_{lit}$ den Literaturwert der Messgröße $x_{mess}$.

\subsection{Zur Bestimmung der Brennweite über die Gegenstands- und Bildweite.}
Zur quantitativen Diskussion der Bestimmung der Brennweite anhand der Gleichung
\eqref{eqn:LG} wurde der relative Fehler der gemittelten Brennweite
\SI{0.0983(4)}{\meter} zur Herstellerangabe von \SI{0.1}{\meter} Brennweite
bestimmt, dieser beträgt \SI{1.7(4)}{\percent}. Ebenfalls wurde der relative
Fehler des Abbildungsmaßstabes bestimmt über die Gegenstands- und Bildweiten
\SI{1.5(6)}{\meter}
zum Abbildungsmaßstabes bestimmt über die Gegenstands- und Bildgröße
\SI{1.4(6)}{\meter} bestimmt, dieser beträgt \SI{0(6)e1}{\percent}.
Im Rahmen der vom Versuchsaufbau gegeben Genauigkeit können die Gleichungen
\eqref{eqn:AG} und \eqref{eqn:LG} als verifiziert angesehen werden.

\subsection{Zur Methode nach Bessel.}
Es wurde zur quantitativen Diskussion der relative Fehler von
der mittleren Brennweite zur Herstellerangabe von \SI{0.1}{\meter} bestimmt.
Für normales Licht wurde eine Brennweite von \SI{0.040(6)}{\meter} bestimmt,
der relative Fehler beträgt also \SI{60(6)}{\percent}.
Für rotes und blaues Licht wurde eine Brennweite von \SI{0.034(6)}{\meter}
bestimmt, der relative Fehler beträgt \SI{66(6)}{\percent}. Die Abweichung
können damit erklärt werden, dass es sehr schwer ist zusagen wann das
Bild wirklich scharf ist asufgrund der sphärischen Abberation.

\subsection{Zur Methode nach Abbe.}
Zur quantitativen Diskussion wurde die Abweichung
\begin{equation*}
  \lvert \bar{f_{g'}} - \bar{f_{b'}} \rvert
  \qquad \text{und} \qquad
  \lvert h - h' \rvert
\end{equation*}
berechnet. Die Abweichung der Brennweiten beträgt \SI{0.09(2)}{\meter}
und die der y-Achsenabschnitte beträgt \SI{0.29(5)}{\meter}. Da die Brennweiten
sich addieren sollte eigentlich keine Brennweite gemessen werden sollen.
Weshalb das Ergebnis daruf zurück zu führen ist, dass in diesem Fall nicht
mehr eindeutig bestimmt worden konnte ob es sich um ein scharfes Bild
handelte.
