\section{Durchführung}
\label{sec:Durchführung}
Zunächst wird für eine Linse mit Brennweite $\SI{100}{\milli \meter}$ für 10 verschiedene Gegenstandsweiten die Bildweite bestimmt, bei der das Bild scharf abgebildet wird, sowie die zugehörige Bildgröße.
Anschließend wird eine Linse verwendet, welche mit Wasser voll gepumpt werden kann, um die Brennweite ein zu stellen. Um die Brennweite konstant zu halten wird die manuelle Pumpe mit einem Gummiband fixiert. Nun wird wie mit der $\SI{100}{\milli \meter}$-Linse verfahren, um die Brennweite zu bestimmen.
Die Methode nach Bessel wird ebenfalls zehnmal durchgeführt für die $\SI{100}{\milli \meter}$-Linse. Zusätzlich wird je fünf mal mit einem roten, bzw. blauen Filter vor der Lichtquelle gemessen, um den Effekt der sphärischen Abberation zu quantitativ zu bestimmen.
Zuletzt wird die Methode nach Abbe verwendet, um für ein Linsensystem aus einer Sammellinse mit $f=\SI{100}{\milli \meter}$ und einer Streulinse mit $f=\SI{-100}{\milli \meter}$ die Gesamtbrennweite zu bestimmen.
