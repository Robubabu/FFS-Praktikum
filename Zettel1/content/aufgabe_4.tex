\section{Aufgabe 4}
\label{sec:a4}
a) \newline
Nach Quantentheorie beugen Elektronen an Spalten wie Wellen und an Doppelspalten interferieren sie mit sich selbst. Ihr Wellencharakter gibt allerdings nur die Wahrscheinlichkeit wo das einzelne Elektron auftrifft an, für "schöne" Beugungsbilder müssen also viele Elektronen gemessen werden.
\newline
\newline
b)\newline
Die Elektronen besitzen die kinetische Energie $E_k = \frac{p^2}{2m_e} \implies p=\sqrt{2m_e E_k}$ Aus der Formel für die de-Broglie-Wellenlänge folgt: $\lambda = \frac{h}{p} = \frac{h}{\sqrt{2m_e E_k}}.$\newline
\newline
c) \newline
Aus der Interferenzbedingung am Doppelspalt folgt: $\alpha = \sin^{-1}\left(\frac{\lambda}{d}\right)= \sin^{-1}\left(\frac{h}{d \sqrt{2m_e E_k}}\right) = \sin^{-1}\left(\frac{h}{d \sqrt{2m_e e_0 U}}\right) = 7 \cdot 10^{-9}°$.
