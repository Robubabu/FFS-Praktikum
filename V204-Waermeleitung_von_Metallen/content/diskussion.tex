\section{Diskussion}
\label{sec:Diskussion}
\subsection{Zur statischen Methode}
Aus der statischen Methode ist ersichtich, dass Aluminium die beste
Wärmeleitfähigkeit von allen untersuchten Metallen hat mit absteigender
Wärmeleitfähigkeit folgen dann Messing und Edelstahl. Auch die Berechnung des
Wärmestroms zeigte, dass Edelstahl den geringsten Wärmestrom hat.
\subsection{Zur dynamischen Methode}
Die dynamische Methode zeigt eindeutig, dass Aluminium die höhere
Wärmeleitfähigkeit hat danach kommt Messing und dann Edelstahl.
Um eine quantitative Aussage über die Qualitat der durchgeführt Messung zu
machen, werden die relativen Fehler $\bar{\kappa}$ mit der Gleichung
\begin{equation*}
  \bar{\kappa} = \frac{\lvert \kappa_e - x_l\rvert}{\lvert x_l\rvert} \cdot 100 \%
\end{equation*}
berechnet. Dabei bezeichnet $\kappa_e$ die experimentel erhobenen
Wärmeleitfähigkeiten und $x_l$ die Literaturwerte. Die Gegenüberstellung
von experimentel erhobenen Werten, Literaturwerten und relativer Fehler ist in
der Tabelle \ref{tab:rf} dargestellt.
\begin{table}
  \centering
  \caption{Wärmeleitfähigkeiten und ihre relativen Fehler}
  \begin{tabular}{c c c c}
    \toprule
    Metal & $\kappa_{experimentel}$/ \si{\watt\per\meter\per\kelvin}
    & $\kappa_{Literatur}$ / \si{\watt\per\meter\per\kelvin}
    & rel. Fehler/\si{\percent} \\
    \midrule
      Messing & 117.35 & 120 & 2.2  \\
      Aluminium & 203.19 & 236 & 13.9 \\
      Edelstahl & 20.72 & 58 & 64.27 \\
    \bottomrule
  \end{tabular}
  \label{tab:rf}
\end{table}
\FloatBarrier
Die relative Abweichung für Messing und Aluminium ist mit algemeinen Messungenauigkeiten,
sowie Verunreinigungen in den Messstäben zu erklären. Die hohe Abweichung bei
Edelstahl lässt sich nur mit einem ungeeignetem Literaturwert erklären.
Denn Edelstahl hat große Unterschiede in der Beschaffenheit und damit große
Unterschiede in seiner Wärmeleitfähigkeit, da (Edel-)Stahl immer ein Gemisch von
Stoffen mit Hauptbestandteil Eisen ist.
