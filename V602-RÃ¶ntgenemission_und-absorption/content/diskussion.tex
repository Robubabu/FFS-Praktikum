\section{Diskussion}
\label{sec:Diskussion}

Bei der Kontrolle der Bragg-Bedingung liegt eine Abweichung von $2 \si{\degree}$ vor, welche durch 2 Fehleinstellungen am Röntgengerät verursacht wird, welche in den nachfolgenden Messungen behoben sind. Der Kristall ist nicht exakt eingehängt worden, weshalb der fest eingestellte Winkel von 14° nicht dem tatsächlichen Winkel zwischen Kristall und Röntgenstrahl entspricht. Zudem wurde das Kabel des Geiger-Müller-Zählrohres so um den Aufbau geschlungen, dass es bei großen Winkeln verkeilt.

Die durch das Emissionspektrum gewonnenen Daten liegen allesamt innerhalb der veranschlagten Auflösung der Apparatur und sind somit sehr exakt bestimmt, lediglich die L-Linien des Quecksilber weichen mit $\Delta E > 0.3$keV etwas stärker ab. Die Abschirmkonstanten geben lediglich einen groben Richtwert, da sie ohnehin nur durch grobe Näherung bestimmt wurde,weshalb diese Werte nicht diskussionswürdig sind.

Die Messung der Rydbergenergie liefert einen um (1.52 $\pm$ 0.11)eV abweichenden Wert, was vor allem darin begründet liegt, dass eine grobe Näherung vorgenommen wurde durch Vernachlässigung der Abschirmung.
