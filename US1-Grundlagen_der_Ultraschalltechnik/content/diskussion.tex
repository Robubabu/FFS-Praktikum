\section{Diskussion}
\label{sec:Diskussion}
\subsection{Methoden}
Alle relativen Fehler wurden nach der Formel
\begin{equation*}
  \tilde{x} = \frac{ \lvert x_{lit} - x_{mess} \rvert}{\lvert x_{lit} \rvert}
  \cdot 100 \%
\end{equation*}
berechnet, dabei bezeichnet $x_{lit}$ den Literaturwert der Messgröße $x_{mess}$.
\subsection{Zur Bestimmung der materialspezifischen Schallgeschwindigkeit}
Der Literaturwert der Schallgeschwindgkeit in Acryl \cite{oly} liegt bei
\SI{2730}{\meter \per \second}. Der relative Fehler liegt bei der Messung mit
dem Impuls-Echo-Verfahren bei \SI{0.4}{\percent} und bei dem Durschallungs-Verfahren
bei \SI{0.5}{\percent}. Der relative Fehler von der gemittelten Schallgeschwindgkeit
liegt bei \SI{0.4}{\percent}. Die Abweichungen entstehen durch Materialunreinheiten
sowie Streuungs-und Absorbtionseffekte.
\subsection{Zur Bestimmung des materialspezifischen Schwächungskoeffizienten}
Der Literaturwert liegt zwischen
\SI{2.7}{\dB\per\centi\meter} bis \SI{5.7}{\dB\per\centi\meter} \cite{gampt} im Frequenzbereich
von \SI{1}{\mega\hertz} bis \SI{4}{\mega\hertz}. Das entspricht
\SI{270}{\dB\per\meter} bis \SI{570}{\dB\per\meter} somit liegt der relative
Fehler bei \SI{7(05)e4}{\percent} bis \SI{3.3(2)e4}{\percent}. Entweder war die
Formel zur Bestimmung des $\alpha$ in \si{\dB\per\meter} inkorrekt oder
es wurden andere Verstärkungen als die der TGC nicht abgerechnet.

\subsection{Spektrale Analyse}
Da die Platten die bei diesem Versuchsteil nocheinmal mit einer Schieblehre
vermessen wurden können nun die Werte verglichen werden. Die erste Platte hatte
mit der Schieblehre gemessen, eine Dicke von \SI{6}{\milli\meter} die zweite
eine Dicke von \SI{1}{\centi\meter}. Daraus ergibt sich der relative Fehler für
die erste Platte mit \SI{10.4}{\percent} und für die zweite Platte mit
\SI{22.5}{\percent}. Die Abweichungen können durch Ablesefehler erklärt werden.
Zudem ist zu erwähnen, dass zur Datengewinnung
auch das Cestrum hinzugenommen werden sollte, da es die Distanz der Echos zu einander
zeigt, aber das bei dem Versuch aufgenommene zeigt nicht genügend Peaks und lieferte
schon im Überschlag zu ungenaue Daten.
\subsection{Biometrische Untersuchung des Augenmodels}
In der Literatur \cite{wiki} ist zu finden, dass die Linsendicke bei einem Erwachsenem
zwischen 4,0 und \SI{4.7}{\milli\meter} liegt, was auch ungefähr den berechneten
Werten des Augenmodells entspricht. Der Durchmesser des Augapfel ist bei
einem Erwachsenen um die \SI{2.4}{\centi\meter} groß, dies müsste ungefähr dem
Linsen Retina Abstand entsprechen. Vergleicht man aber die Werte so ist eine
starke Abweichung zu erkennen. Abweichungen können hier durch Ablesefehler aber
auch durch die Tatsache, dass biometrische Daten nicht absolut und exakt sind erklärt werden.
Zudem steht das Modell im Maßstab 1:3 zu einem menschlichen Auge.
