\section{Durchführung}
\label{sec:Durchführung}
In diesem Versuch werden zunächst Acrylzylinder einmal mit dem Durchschallungs- und einmal mit dem Impuls-Echo-Verfahren vermessen. Hierzu werden für x Zylinder die Schalllaufzeiten aufgezeichnet. Beim Impuls-Echo-Verfahren werden zusätzlich die Amplituden der Impulse gemessen, um den exponentiellen Abfall dieser nachzuweisen, wobei eine feste Verstärkung des Signales (engl. gain), sowie eine zeitlich variable Verstärkung (engl. time gain control, kurz TGC) zur Verfügung stehen, um die Signale zu verdeutlichen. Da beim Impuls-Echo-Verfahren die Acrylzylinder senkrecht stehen können, um mehr Messdaten zu erhalten, mehrere Zylinder (mit bidestilliertem Wasser gekoppelt) aufeinander gestellt werden. Alle Zylinder werden mit Hilfe einer Schieblehre vermessen, um aus ihrer Länge und der aufgenommenen Laufzeiten die Schallgeschwindigkeit in Acryl zu berechnen.
Nach Bestimmung der Schallgeschwindigkeit werden mit dem Impuls-Echo-Verfahren Grenzflächen untersucht: zwei dünne Acrylscheiben werden übereinander gekoppelt und über einen Vorlaufzylinder beschallt. Der Vorlaufzylinder besteht ebenfalls aus Acryl und sorgt nur dafür, dass die reflektierten Impulse (welche sehr Schwach ausfallen) gut erkennbar vom ersten Impuls zu unterscheiden sind.
Abschließend verwendet man das Puls-Echo-Verfahren, um eine Augatrappe mit Iris und Linse zu vermessen.
