\section{Auswertung}
\label{sec:Auswertung}
\subsection{Bestimmung der materialspezifischen Schallgeschwindigkeit}
Zur Bestimmung der materialspezifischen Schallgeschwindigkeit werden die
Durchlaufzeiten gegen die durchlaufene Strecke aufgetragen. Für das
Impuls-Echo-Verfahren gilt der Zusammenhang
\begin{equation*}
  h = \frac{1}{2} ct
\end{equation*}
für das Durchschallung-Verfahren der Zusammenhang
\begin{equation*}
  h = ct \; .
\end{equation*}
Aus den aufgetragenen Werten kann dann für das Impuls-Echo-Verfahren
eine Ausgleichsgerade mit der Form
\begin{equation*}
  t(h) = 2 \frac{h}{c} + \Delta t
\end{equation*}
berechnet werden. Für das Durchschallungsverfahren ergibt sich eine
Ausgleichgerade mit der Form
\begin{equation*}
  t(h) = \frac{h}{c} + \Delta t \; .
\end{equation*}
Alle Werte sind in den Tabellen \ref{tab:Tabs} dargestellt. Die Ausgleichsgeraden
mit Werten sind in den Abbildungen \ref{fig:IEplot} und \ref{fig:DSplot} dargestellt.
Der Ausdruck $ \Delta t$ in den Ausgleichsgeraden bezeichnet den systematischen
Fehler der aufgrund der Anpassungsschicht der Sonden entsteht. Aus der
Ausgleichsrechnung für das Impuls-Echo-Vefahren lässt sich dann entnehmen,
dass die materialspezifische Schallgeschwindigkeit
\begin{equation*}
  c_{IE} = \SI{2720(10)}{\meter\per\second}
\end{equation*}
und der systematische Fehler
\begin{equation*}
  \Delta t = \SI{3(2)e-7}{\second}
\end{equation*}
beträgt. Für das Durchschallungs-Verfahren ergeben sich die Werte
\begin{equation*}
  c_{DS} = \SI{2717(31)}{\meter \per \second}
\end{equation*}
für die Schallgeschwindigkeit und für den systematischen Fehler
\begin{equation*}
  \Delta t = \SI{1.3 (3)e-6}{\second}\; .
\end{equation*}
Im Mittel ergibt sich also für die Schallgeschwindigkeit in Acryl
\begin{equation*}
  \bar{c} = \SI{2718(16)}{\meter \per \second}
\end{equation*}
\begin{figure}
  \centering
  \caption{Werte des . . . }
  \begin{subfigure}{0.48\textwidth}
    \caption{. . . Impuls-Echo-Verfahren im Überblick.}
    \sisetup{round-mode = places, round-precision = 2}
    \begin{tabular}{S S}
      \toprule
      h / \si{\meter} & t / \si{\second} \\
      \midrule
      1.200600000000000001e-01 & 8.770000000000000397e-05\\
      1.020000000000000073e-01 & 7.506000000000000302e-05\\
      8.009999999999999065e-02 & 5.846999999999999415e-05\\
      4.003999999999999920e-02 & 2.907999999999999602e-05\\
      3.099999999999999978e-02 & 2.243999999999999860e-05\\
      1.110999999999999904e-01 & 8.137999999999999672e-05\\
      7.104000000000000592e-02 & 5.214999999999999367e-05\\
      \bottomrule
    \end{tabular}
    \label{tab:IE}
  \end{subfigure}
  \begin{subfigure}{0.48\textwidth}
    \caption{. . . Durchschallungs-Verfahren im Überblick.}
    \sisetup{round-mode= places, round-precision = 2}
    \begin{tabular}{S S}
      \toprule
      h / \si{\meter} & t / \si{\second} \\
      \midrule
      1.200600000000000001e-01 & 4.518999999999999254e-05\\
      1.020000000000000073e-01 & 3.902999999999999731e-05\\
      8.009999999999999065e-02 & 3.080999999999999809e-05\\
      4.003999999999999920e-02 & 1.627999999999999998e-05\\
      3.099999999999999978e-02 & 1.232999999999999884e-05\\
      \bottomrule
    \end{tabular}
    \label{tab:DS}
  \end{subfigure}
  \label{tab:Tabs}
\end{figure}




\begin{figure}
  \centering
  \includegraphics[height= 7cm]{plots/IE_plot.pdf}
  \caption{Impuls-Echo-Verfahren}
  \label{fig:IEplot}
\end{figure}
\begin{figure}
  \centering
  \includegraphics[height = 7cm]{plots/DS_plot.pdf}
  \caption{Durchsschallungs-Verfahren}
  \label{fig:DSplot}
\end{figure}
\subsection{Spektrale Analyse}
Duch den Zusammenhang
\begin{equation*}
h = \frac{c}{2} t
\end{equation*}
können die Dicken der Blatten die mit dem Impuls-Echo-Verfahren durchleuchtet
wurden bestimmt werden. Aus der Abbildung \ref{fig:SA} werden die Positionen
der Echopeaks entnommen (s.Tabelle \ref{tab:SAW}), welche anzeigen wo bzw. wann der Ultraschall eine
Grenzfläche passiert. Somit verändert sich der Zusammenhang wie folgt:
\begin{equation*}
  h = \frac{\bar{c}}{2} \Delta t \; .
\end{equation*}
Das $\Delta t $ bezeichnet dabei den Zeitunterschied zwischen den Peaks.
Daraus ergibt sich dann, dass die erste Platte ca. \SI{5.43(3)}{\milli\meter}
und die zweite Platte ca. \SI{8.16(5)}{\milli\meter} dick ist.
\begin{table}
  \centering
  \caption{Aus Abbildung \ref{fig:SA} entnommene Werte}
  \begin{tabular}{c c}
    \toprule
    Peak & Depth  / \si{\micro\second}\\
    \midrule
    1 & 2 \\
    2 & 30 \\
    3 & 34 \\
    4 & 40 \\
    \bottomrule
  \end{tabular}
\label{tab:SAW}
\end{table}


\begin{figure}
  \centering
  \includegraphics[height = 5cm]{FFS-Data/a_scan_spektrale_analyse.jpg}
  \caption{Spektrale Analyse}
  \label{fig:SA}
\end{figure}
% \begin{figure}
%   \centering
%   \includegraphics[height = 5cm]{FFS-Data/cepstrum_spektrale_analyse.jpg}
%   \label{fig:SAC}
% \end{figure}
\subsection{Biometrische Untersuchung des Augenmodells}
Wie im Kapitel zuvor wird aus der Abbildung \ref{fig:AU} die Positionen der
Peaks bestimmt (s. Tabelle \ref{tab:AU}) und mit dem selben Zusammenhang wie zuvor die Abstände innerhalb
des Auges bestimmt. Hierbei ist doch zubeachten, dass es innerhalb des Auges
verschiedene Schallgeschwindgkeiten gibt. So beträgt die Schallgeschwindgkeit
in der Linse \SI{2500}{\meter\per\second} und in der Glaskörperflüssigkeit
\SI{1410}{\meter \per \second}.
Daraus ergibt sich dann für den Abstand zwischen Hornhaut und Linse
\SI{5.6}{\milli\meter} und für den Durchmesser der Linse \SI{3.7}{\milli\meter}.
Der Abstand zwischen Linse und Retina beträgt \SI{3.8}{\centi\meter}.

\begin{table}
  \centering
  \caption{Aus Abbildung \ref{fig:AU} entnommene Daten}
  \begin{tabular}{c c}
  \toprule
  Peak & Depth /\si{\micro \second} \\
  \midrule
  1 & 2 \\
  2 & 10 \\
  3 & 13 \\
  4 & 67 \\
  \bottomrule
\end{tabular}
\label{tab:AU}
\end{table}

\begin{figure}
  \centering
  \includegraphics[height = 5cm]{FFS-Data/a_scan_auge.jpg}
  \caption{Spektrale Analyse}
  \label{fig:AU}
\end{figure}
