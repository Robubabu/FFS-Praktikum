\section{Diskussion}
\label{sec:Diskussion}
\subsection{Langmuir-Schottkysches Gesetz}
Das Langmuir-Schottkysche Gesetz sieht einen Spannungsexponenten von $a = 3/2$ vor, experimentell ergibt sich $a = 1.9 \pm 0.1$. Da der Gültigkeitsbereich des Gesetzes geschätzt wurde und nur wenige Messwerte zur Verfügung stehen, stellt dieser Wert ein relativ gutes Ergebnis dar.

\subsection{Anlaufstrom und Kathodentemperatur}
Die über den Ausgleich des Anlaufstromes berechnete Kathodentemperatur beträgt $T = \SI{2050 \pm 200}{\kelvin}$, was gut zu den über die Leistungsbilanz berechneten $\SI{2057}{\kelvin}$ passt, allerdings mit eine enorme Ungenauigkeit behaftet ist. Ansonsten liegen die Temperaturen für thermische Elektronenemission in einem realistischen Bereich.

\subsection{Austrittsarbeit}
Eine Austrittsarbeit von $\phi =  \SI{2.45 \pm 0.06}{\electronvolt}$ ist viel zu gering für Wolfram. Akzeptabel wären Werte zwischen $\SI{4}{\electronvolt}$ und $\SI{5}{\electronvolt}$. Vermutlich wurde ein Fehler bei der Berechnung der Temperaturen begangen.
