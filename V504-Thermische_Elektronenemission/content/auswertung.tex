\section{Auswertung}
\label{sec:Auswertung}
\subsection{Kennlinien der Diode}
Die aufgenommenen Werte lassen sich Tab. \ref{tab:Reihe1}, \ref{tab:Reihe2}, \ref{tab:Reihe3}, \ref{tab:Reihe4}, \ref{tab:Reihe4}, sowie Abb. \ref{fig:Plot1}, \ref{fig:Plot2}, \ref{fig:Plot3}, \ref{fig:Plot4} \ref{fig:Plot5}, entnehmen. Die Sättigungsströme ergeben sich so zu
\begin{align*}
  I_{S,1} &= \SI{0.012}{\ampere} \\
  I_{S,2} &= \SI{0.024}{\ampere} \\
  I_{S,3} &= \SI{0.081}{\ampere} \\
  I_{S,4} &= \SI{0.146}{\ampere} \\
  I_{S,5} &= \SI{0.342}{\ampere}
\end{align*}
\begin{figure}
  \centering
  \includegraphics[height=7cm]{./plots/Plot1.pdf}
  \caption{Kennlinie der Diode, aufgenommen bei $U_h = \SI{3.5}{\volt}$, $I_h = \SI{1.8}{\ampere}$}
  \label{fig:Plot1}
\end{figure}

\begin{figure}
  \centering
  \includegraphics[height=7cm]{./plots/Plot2.pdf}
  \caption{Kennlinie der Diode, aufgenommen bei $U_h = \SI{4}{\volt}$, $I_h = \SI{1.9}{\ampere}$}
  \label{fig:Plot2}
\end{figure}

\begin{figure}
  \centering
  \includegraphics[height=7cm]{./plots/Plot3.pdf}
  \caption{Kennlinie der Diode, aufgenommen bei $U_h = \SI{4.5}{\volt}$, $I_h = \SI{2}{\ampere}$}
  \label{fig:Plot3}
\end{figure}

\begin{figure}
  \centering
  \includegraphics[height=7cm]{./plots/Plot4.pdf}
  \caption{Kennlinie der Diode, aufgenommen bei $U_h = \SI{4.5}{\volt}$, $I_h = \SI{2.1}{\ampere}$}
  \label{fig:Plot4}
\end{figure}

\begin{figure}
  \centering
  \includegraphics[height=7cm]{./plots/Plot5.pdf}
  \caption{Kennlinie der Diode, aufgenommen bei $U_h = \SI{5}{\volt}$, $I_h = \SI{2.2}{\ampere}$}
  \label{fig:Plot5}
\end{figure}

\begin{table}
  \centering
  \sisetup{round-mode = places , round-precision = 2}
  \caption{Messwerte der ersten Messreihe}
  \label{tab:Reihe1}
  \begin{tabular}{|S|S|}
    \toprule
    $U/\si{\volt}$ & $I/\si{\ampere}$ \\
    \midrule
    0.000000000000000000e+00 & 0.000000000000000000e+00\\
    5.000000000000000000e+00 & 1.000000000000000021e-03\\
    1.000000000000000000e+01 & 6.000000000000000125e-03\\
    1.500000000000000000e+01 & 8.000000000000000167e-03\\
    2.000000000000000000e+01 & 1.000000000000000021e-02\\
    2.500000000000000000e+01 & 1.000000000000000021e-02\\
    3.000000000000000000e+01 & 1.099999999999999936e-02\\
    3.500000000000000000e+01 & 1.099999999999999936e-02\\
    4.000000000000000000e+01 & 1.099999999999999936e-02\\
    4.500000000000000000e+01 & 1.099999999999999936e-02\\
    5.000000000000000000e+01 & 1.099999999999999936e-02\\
    5.500000000000000000e+01 & 1.099999999999999936e-02\\
    6.000000000000000000e+01 & 1.099999999999999936e-02\\
    6.500000000000000000e+01 & 1.099999999999999936e-02\\
    7.000000000000000000e+01 & 1.099999999999999936e-02\\
    7.500000000000000000e+01 & 1.200000000000000025e-02\\
    8.000000000000000000e+01 & 1.200000000000000025e-02\\
    8.500000000000000000e+01 & 1.200000000000000025e-02\\
    9.000000000000000000e+01 & 1.099999999999999936e-02\\
    9.500000000000000000e+01 & 1.200000000000000025e-02\\
    1.000000000000000000e+02 & 1.200000000000000025e-02\\
    1.050000000000000000e+02 & 1.200000000000000025e-02\\
    1.100000000000000000e+02 & 1.200000000000000025e-02\\
    1.150000000000000000e+02 & 1.200000000000000025e-02\\
    1.200000000000000000e+02 & 1.200000000000000025e-02\\
    1.300000000000000000e+02 & 1.200000000000000025e-02\\
    1.400000000000000000e+02 & 1.200000000000000025e-02\\
    1.500000000000000000e+02 & 1.200000000000000025e-02\\
    1.600000000000000000e+02 & 1.200000000000000025e-02\\
    1.700000000000000000e+02 & 1.200000000000000025e-02\\
    1.800000000000000000e+02 & 1.200000000000000025e-02\\
    1.900000000000000000e+02 & 1.200000000000000025e-02\\
    \bottomrule
  \end{tabular}
\end{table}

\begin{table}
  \centering
  \sisetup{round-mode = places , round-precision = 2}
  \caption{Messwerte der zweiten Messreihe}
  \label{tab:Reihe2}
  \begin{tabular}{|S|S|}
    \toprule
    $U/\si{\volt}$ & $I/\si{\ampere}$ \\
    \midrule
    0.000000000000000000e+00 & 0.000000000000000000e+00\\
    5.000000000000000000e+00 & 0.000000000000000000e+00\\
    1.000000000000000000e+01 & 5.000000000000000104e-03\\
    1.500000000000000000e+01 & 1.299999999999999940e-02\\
    2.000000000000000000e+01 & 1.799999999999999864e-02\\
    2.500000000000000000e+01 & 2.100000000000000130e-02\\
    3.000000000000000000e+01 & 2.100000000000000130e-02\\
    3.500000000000000000e+01 & 2.199999999999999872e-02\\
    4.000000000000000000e+01 & 2.199999999999999872e-02\\
    4.500000000000000000e+01 & 2.299999999999999961e-02\\
    5.000000000000000000e+01 & 2.299999999999999961e-02\\
    6.000000000000000000e+01 & 2.299999999999999961e-02\\
    6.500000000000000000e+01 & 2.299999999999999961e-02\\
    7.000000000000000000e+01 & 2.299999999999999961e-02\\
    7.500000000000000000e+01 & 2.299999999999999961e-02\\
    8.000000000000000000e+01 & 2.400000000000000050e-02\\
    8.500000000000000000e+01 & 2.299999999999999961e-02\\
    9.000000000000000000e+01 & 2.299999999999999961e-02\\
    9.500000000000000000e+01 & 2.400000000000000050e-02\\
    1.000000000000000000e+02 & 2.400000000000000050e-02\\
    1.050000000000000000e+02 & 2.400000000000000050e-02\\
    \bottomrule
  \end{tabular}
\end{table}

\begin{table}
  \centering
  \sisetup{round-mode = places , round-precision = 2}
  \caption{Messwerte der dritten Messreihe}
  \label{tab:Reihe3}
  \begin{tabular}{|S|S|}
    \toprule
    $U/\si{\volt}$ & $I/\si{\ampere}$ \\
    \midrule
    0.000000000000000000e+00 & 0.000000000000000000e+00\\
    5.000000000000000000e+00 & 1.000000000000000021e-03\\
    1.000000000000000000e+01 & 1.499999999999999944e-02\\
    1.500000000000000000e+01 & 3.599999999999999728e-02\\
    2.000000000000000000e+01 & 5.399999999999999939e-02\\
    2.500000000000000000e+01 & 6.400000000000000133e-02\\
    3.000000000000000000e+01 & 6.900000000000000577e-02\\
    3.500000000000000000e+01 & 7.099999999999999367e-02\\
    4.000000000000000000e+01 & 7.299999999999999545e-02\\
    4.500000000000000000e+01 & 7.299999999999999545e-02\\
    5.000000000000000000e+01 & 7.399999999999999634e-02\\
    5.500000000000000000e+01 & 7.499999999999999722e-02\\
    6.000000000000000000e+01 & 7.499999999999999722e-02\\
    6.500000000000000000e+01 & 7.599999999999999811e-02\\
    7.000000000000000000e+01 & 7.599999999999999811e-02\\
    7.500000000000000000e+01 & 7.599999999999999811e-02\\
    8.000000000000000000e+01 & 7.699999999999999900e-02\\
    8.500000000000000000e+01 & 7.699999999999999900e-02\\
    9.000000000000000000e+01 & 7.699999999999999900e-02\\
    9.500000000000000000e+01 & 7.699999999999999900e-02\\
    1.000000000000000000e+02 & 7.799999999999999989e-02\\
    1.050000000000000000e+02 & 7.900000000000000078e-02\\
    1.100000000000000000e+02 & 7.799999999999999989e-02\\
    1.200000000000000000e+02 & 7.799999999999999989e-02\\
    1.300000000000000000e+02 & 7.900000000000000078e-02\\
    1.400000000000000000e+02 & 7.900000000000000078e-02\\
    1.500000000000000000e+02 & 7.900000000000000078e-02\\
    1.600000000000000000e+02 & 7.900000000000000078e-02\\
    1.700000000000000000e+02 & 8.000000000000000167e-02\\
    1.800000000000000000e+02 & 8.000000000000000167e-02\\
    1.900000000000000000e+02 & 8.000000000000000167e-02\\
    2.000000000000000000e+02 & 8.100000000000000255e-02\\
    2.100000000000000000e+02 & 8.100000000000000255e-02\\
    2.200000000000000000e+02 & 8.100000000000000255e-02\\
    2.300000000000000000e+02 & 8.100000000000000255e-02\\
    2.400000000000000000e+02 & 8.100000000000000255e-02\\
    2.500000000000000000e+02 & 8.100000000000000255e-02\\
    \bottomrule
  \end{tabular}
\end{table}

\begin{table}
  \centering
  \sisetup{round-mode = places , round-precision = 2}
  \caption{Messwerte der vierten Messreihe}
  \label{tab:Reihe4}
  \begin{tabular}{|S|S|}
    \toprule
    $U/\si{\volt}$ & $I/\si{\ampere}$ \\
    \midrule
    0.000000000000000000e+00 & 0.000000000000000000e+00\\
    5.000000000000000000e+00 & 0.000000000000000000e+00\\
    1.000000000000000000e+01 & 4.199999999999999740e-03\\
    1.500000000000000000e+01 & 2.299999999999999961e-02\\
    2.000000000000000000e+01 & 5.299999999999999850e-02\\
    2.500000000000000000e+01 & 8.999999999999999667e-02\\
    3.000000000000000000e+01 & 1.150000000000000050e-01\\
    3.500000000000000000e+01 & 1.290000000000000036e-01\\
    4.000000000000000000e+01 & 1.350000000000000089e-01\\
    4.500000000000000000e+01 & 1.380000000000000115e-01\\
    5.000000000000000000e+01 & 1.400000000000000133e-01\\
    5.500000000000000000e+01 & 1.409999999999999865e-01\\
    6.000000000000000000e+01 & 1.419999999999999873e-01\\
    6.500000000000000000e+01 & 1.419999999999999873e-01\\
    7.000000000000000000e+01 & 1.429999999999999882e-01\\
    7.500000000000000000e+01 & 1.439999999999999891e-01\\
    8.000000000000000000e+01 & 1.439999999999999891e-01\\
    8.500000000000000000e+01 & 1.439999999999999891e-01\\
    9.000000000000000000e+01 & 1.449999999999999900e-01\\
    1.000000000000000000e+02 & 1.459999999999999909e-01\\
    \bottomrule
  \end{tabular}
\end{table}

\begin{table}
  \centering
  \sisetup{round-mode = places , round-precision = 2}
  \caption{Messwerte der fünften Messreihe}
  \label{tab:Reihe5}
  \begin{tabular}{|S|S|}
    \toprule
    $U/\si{\volt}$ & $I/\si{\ampere}$ \\
    \midrule
    5.000000000000000000e+00 & 8.999999999999999320e-03\\
    1.000000000000000000e+01 & 4.399999999999999745e-02\\
    1.500000000000000000e+01 & 9.400000000000000022e-02\\
    2.000000000000000000e+01 & 1.429999999999999882e-01\\
    2.500000000000000000e+01 & 2.109999999999999931e-01\\
    3.000000000000000000e+01 & 2.469999999999999973e-01\\
    3.500000000000000000e+01 & 2.819999999999999729e-01\\
    4.000000000000000000e+01 & 3.019999999999999907e-01\\
    4.500000000000000000e+01 & 3.140000000000000013e-01\\
    5.000000000000000000e+01 & 3.210000000000000075e-01\\
    5.500000000000000000e+01 & 3.270000000000000129e-01\\
    6.000000000000000000e+01 & 3.300000000000000155e-01\\
    6.500000000000000000e+01 & 3.370000000000000218e-01\\
    7.000000000000000000e+01 & 3.370000000000000218e-01\\
    7.500000000000000000e+01 & 3.380000000000000226e-01\\
    8.000000000000000000e+01 & 3.380000000000000226e-01\\
    8.500000000000000000e+01 & 3.390000000000000235e-01\\
    9.000000000000000000e+01 & 3.410000000000000253e-01\\
    9.500000000000000000e+01 & 3.420000000000000262e-01\\
    -1.000000000000000021e-02 & 3.000000000000000000e+00\\
    -1.000000000000000056e-01 & 3.100000000000000089e+00\\
    -2.000000000000000111e-01 & 2.399999999999999911e+00\\
    -2.999999999999999889e-01 & 1.600000000000000089e+00\\
    -4.000000000000000222e-01 & 1.399999999999999911e+00\\
    -5.000000000000000000e-01 & 1.250000000000000000e+00\\
    -5.999999999999999778e-01 & 8.100000000000000533e-01\\
    -6.999999999999999556e-01 & 5.699999999999999512e-01\\
    -8.000000000000000444e-01 & 2.999999999999999889e-01\\
    -9.000000000000000222e-01 & 1.600000000000000033e-02\\
    -1.000000000000000000e+00 & 8.099999999999999561e-03\\
    \bottomrule
  \end{tabular}
\end{table}
\FloatBarrier

\subsection{Langmuir-Schottkysches Gesetz}

\begin{figure}
  \centering
  \includegraphics[height=7cm]{./plots/langmuir.pdf}
  \caption{Doppellogaritmische Darstellung der $U_h = \SI{5}{\volt}$, $I_h = \SI{2.2}{\ampere}$-Kennlinie, mit linearem Ausgleich}
  \label{fig:langmuir}
\end{figure}

Die Grenze des Raumladungsgebietes wurde auf $\SI{25}{\volt}$ geschätzt, wodurch sich über die in Abb. \ref{fig:langmuir} dargestellte lineare Ausgleichsrechnung
\begin{align*}
  a &= 1.9 \pm  0.1 \\
  b &= -7.6 \pm 0.3]
\end{align*}
mit $a$ als Steigung und $b$  als y-Achsenabschnitt.

\subsection{Anlaufstrom und Kathodentemperatur}
 \begin{figure}
   \centering
   \includegraphics[height = 7cm]{./plots/Plot6.pdf}
   \caption{Logarithmische Darstellung des Anlaufstromes, mit linearem Ausgleich}
   \label{fig:Plot6}
 \end{figure}

Linearer Ausgleich des Anlaufstromes (Abb. \ref{fig:Plot6}) liefert die Werte
\begin{align*}
  a &= 2.8 \pm 0.3 \\
  b &= 1.4 \pm 0.1 \;.
\end{align*}
Aus $a = - \frac{e_0 V}{kT}$ folgt mit $k = \SI{8.7e-5}{\electronvolt \kelvin}$:
\begin{equation*}
  T = \frac{e_0}{k \; a} = \SI{6120 \pm 320}{\kelvin}.
\end{equation*}
Die verwendete Heizleistung beträgt $p = U I = \SI{24.2}{\watt}$.
Die Leistungsbilanz und diese Referenztemperatur liefern
\begin{align*}
  T_1 &= \SI{1830}{\kelvin} \\
  T_2 &= \SI{1920}{\kelvin} \\
  T_3 &= \SI{2000}{\kelvin} \\
  T_4 &= \SI{2030}{\kelvin}.
\end{align*}
Die Standardabweichungen sind hierbei vernachlässigbar klein.

\subsection{Austrittsarbeit der Kathode}

 Über die Richardson Gleichung ergeben sich mit den Temperaturen der Kathode in den ersten 4 Messreihen die äußeren Potentiale
 \begin{align*}
   \phi_1 &= \SI{2.25}{\electronvolt} \\
   \phi_2 &= \SI{2.37}{\electronvolt} \\
   \phi_3 &= \SI{2.48}{\electronvolt} \\
   \phi_4 &= \SI{2.52}{\electronvolt}
 \end{align*}
Mittelung ergibt $\phi =  \SI{2.41 \pm 0.06}{\electronvolt}$.
