\section{Diskussion}
\label{sec:Diskussion}

\subsection{Zu \ref{sec:3.2}}
Die errechneten Werte sind nah an den experimentel ermittelten Werten.
Hier können die Abweichungen durch eine nicht eindeutig bestimmte Amplitude und
systeminterne Widerstände und Messungenauigkeite erklärt werden. Der Fit
ist mit der Gleichung \eqref{eqn:fit} relativ gut
 zu realisieren. \\
 Bei der Sägezahnspannung sind sowohl die errechneten Werte als auch die gemessenen
 Werte sehr nah beieinander. Hier lässt sich die Abweichung durch eine nicht eindeutig
 bestimmte Amplitude so wie systeminterne Wiederstände und
 Messungenauigkeiten zu erklären. Auch der Fit ist hier sehr gelungen. Trotdem
 ist aus der Abbildung \ref{fig:dp} zu lesen, dass wir wohl einen Vorzeichenfehler
  bei unserer Berechnung hatten. \\
 Der Nadelimpuls weicht sehr stark von den experimentellen Werten ab. Auch hier sind die
  Abweichung zwischen gerechneten und gemessenen Werten durch eine nicht
  eindeutige Amplitude zu erklären. Zudem ist zu bemerken, dass die
  Approximation nur für große k gelingt und der Nadelimpuls bei der Durchführung
  des Experimentes nur durch einen sehr schmalen Rechteckimpuls moduliert wurde.

\subsection{Zu \ref{sec:3.3}}
Bei der synthetisierten Rechteck- und Sägezahnspannung können die Abweichungen
dadurch erklärt werden, dass hier nur zehn Oberschwigungen summiert werden.
Der Nadelimpuls stellt sich als sehr schwer zu synthetisieren dar. Hier braucht
man anscheinend wesentlich mehr Komponenten zur Synthese, da es sonst zur Nulldurchläufen
kommt.
