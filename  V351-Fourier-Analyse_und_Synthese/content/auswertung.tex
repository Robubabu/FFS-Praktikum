\section{Auswertung}
\label{sec:Auswertung}
\subsection{Brechnung der Fourier-Koeffizenten}
\subsubsection{Rechteckspannung}
  Für die Rechteckspannung ergibt sich dann die Fourier-Amplituden mit der Fuktion:
  \begin{equation}
    b_n = \frac{A}{n\pi}(1-cos(n\pi))
  \end{equation}
  Die Werte sind in Tabelle \ref{fig:kr} dargestellt für $A =1$.
  \begin{table}
    \centering
    \caption{Fourier-Koeffizenten einer Rechteckspannung}
    \label{fig:kr}
    \sisetup{round-mode = places , round-precision = 2}
    \begin{tabular}{S[round-precision = 0] S}
      \toprule
       \text{Nr.} & \text{Fourier-Amplitude} \\
      \midrule
      1.000000000000000000e+00 & 6.366197723675813824e-01\\
      2.000000000000000000e+00 & 0.000000000000000000e+00\\
      3.000000000000000000e+00 & 2.122065907891937941e-01\\
      4.000000000000000000e+00 & 0.000000000000000000e+00\\
      5.000000000000000000e+00 & 1.273239544735162709e-01\\
      6.000000000000000000e+00 & 0.000000000000000000e+00\\
      7.000000000000000000e+00 & 9.094568176679733440e-02\\
      8.000000000000000000e+00 & 0.000000000000000000e+00\\
      9.000000000000000000e+00 & 7.073553026306460267e-02\\
      \bottomrule
      \end{tabular}
  \end{table}

\subsubsection{Dreickspannung}
Die Fourie-Koeffizenten der Dreieckspannung lassen sich dann durch folgende Gleichung leicht brechnen:
\begin{equation}
  b_n = -\frac{A}{n\pi}
\end{equation}
Die Werte der Fourier-Koeffizenten finden sie in Tabelle \ref{fig:kd} dargestellt für $A = 1$.
\begin{table}
  \centering
  \caption{Fourier-Koeffizenten einer Dreieckspannung}
  \label{fig:kd}
  \sisetup{round-mode = places , round-precision = 2}
  \begin{tabular}{S[round-precision = 0] S}
    \toprule
     \text{Nr.} & \text{Fourier-Amplitude} \\
    \midrule
    1.000000000000000000e+00 & -3.183098861837906912e-01\\
    2.000000000000000000e+00 & -1.591549430918953456e-01\\
    3.000000000000000000e+00 & -1.061032953945968971e-01\\
    4.000000000000000000e+00 & -7.957747154594767280e-02\\
    5.000000000000000000e+00 & -6.366197723675813547e-02\\
    6.000000000000000000e+00 & -5.305164769729844854e-02\\
    7.000000000000000000e+00 & -4.547284088339866720e-02\\
    8.000000000000000000e+00 & -3.978873577297383640e-02\\
    9.000000000000000000e+00 & -3.536776513153230134e-02\\
    \bottomrule
  \end{tabular}
\end{table}
\FloatBarrier
\subsubsection{Nadelimpuls}
  Die Fourier-Koeffizenten des Nadelimpuls werden durch die folgende Gleichung berechnet:
  \begin{equation}
    b_n = \frac{A}{n\pi} sin\left(\frac{n2\pi}{k}\right)
  \end{equation}
  Die Werte finden sie in Tabelle \ref{fig:kn} für verschiedene Werte von k dargestellt mit $A=1$.
\begin{table}
  \centering
  \caption{Fourier-Koeffizenten eines Nadelimpulses}
  \label{fig:kn}
  \begin{subfigure}{0.48\textwidth}
    \centering
  \label{fig:kn10}
  \sisetup{round-mode = places , round-precision = 2}
  \begin{tabular}{S[round-precision = 0] S}
    \toprule
     \text{Nr.} & \text{Fourier-Amplitude} \\
    \midrule
    1.000000000000000000e+00 & 1.870978567577278318e-01\\
    2.000000000000000000e+00 & 1.513653457281314008e-01\\
    3.000000000000000000e+00 & 1.009102304854209431e-01\\
    4.000000000000000000e+00 & 4.677446418943196488e-02\\
    5.000000000000000000e+00 & 7.796343665038751913e-18\\
    6.000000000000000000e+00 & -3.118297612628796386e-02\\
    7.000000000000000000e+00 & -4.324724163660896570e-02\\
    8.000000000000000000e+00 & -3.784133643203285713e-02\\
    9.000000000000000000e+00 & -2.078865075085865530e-02\\
    \bottomrule
  \end{tabular}
  \caption{mit $k = 10$}
\end{subfigure}
  \begin{subfigure}{0.48\textwidth}
    \centering
  \label{fig:kn100}
  \sisetup{round-mode = places , round-precision = 2}
  \begin{tabular}{S[round-precision = 0] S}
    \toprule
     \text{Nr.} & \text{Fourier-Amplitude} \\
    \midrule
    1.000000000000000000e+00 & 1.998684312479682632e-02\\
    2.000000000000000000e+00 & 1.994740365545007166e-02\\
    3.000000000000000000e+00 & 1.988177497291702608e-02\\
    4.000000000000000000e+00 & 1.979011241962616921e-02\\
    5.000000000000000000e+00 & 1.967263286166931816e-02\\
    6.000000000000000000e+00 & 1.952961407775311367e-02\\
    7.000000000000000000e+00 & 1.936139397678475829e-02\\
    8.000000000000000000e+00 & 1.916836964649249950e-02\\
    9.000000000000000000e+00 & 1.895099623599886401e-02\\
    \bottomrule
  \end{tabular}
  \caption{mit $k = 100$}
\end{subfigure}
\end{table}
\FloatBarrier
Aus Tabelle \ref{fig:kn}\subref{fig:kn100} ist ersichtlich, dass sich für hohe Werte von k ein sog. Dirac-Kamm ergibt.
\subsection{Vergleich mit den experimentel erfassten Fourier-Koeffizenten}
\subsubsection{Rechteckspannung}
In Abbildung \ref{fig:pr} sind die rechnerisch und experimentel ermittelten Fourier-Koeffizenten aufgetragen. Die experimentellen Größen wurden alle mit einem Fehler
von 3\% versehen, da sie von einem Oszilloskop abgelesen wurden. Die Amplitude A der brechnetnen Werte wurde hier mit 10 Volt angenährt.
Dennoch ist zu beobachten, dass die Erwartungswerte nicht innerhalb der Fehlertoleranz liegen.
\begin{figure}
  \centering
  \includegraphics{Rechteckplot.pdf}
  \caption{Rechnerische und experimentelle Fourier-Amplituden der Rechteckspannung}
  \label{fig:pr}
\end{figure}
\FloatBarrier
\subsubsection{Dreiecksspannung}
Auch hier wurden die rechnerisch und experimentel ermittelten Fourier-Koeffizenten in
ein Diagramm eingetragen (s.Abb.\ref{fig:dp}), wieder wurden
die experimentellen Größen mit einem Fehler von 3\% versehen und die Amplitude A mit 10 Volt
angenommen.
Zudem wurden die Erwartungswerte mit anderem Vorzeichen eingezeichnet, damit sich die
 Kurve den experimentellen Werten angleicht.
Werden nur die Erwatungswerte mit anderem Vorzeichen und die experimentellen Werte verglichen
 ist zu sehen, dass hier die Werte nur am Anfang der Messreihe außerhalb der Fehlertoleranz liegen.
\begin{figure}
  \centering
  \includegraphics{Dreieckplot.pdf}
  \caption{Rechnerische und experimentelle Fourier-Amplituden der Dreieckspannung}
  \label{fig:dp}
\end{figure}
\FloatBarrier
\subsubsection{Nadelimpuls}
Wie auch in den beiden Messreihen davor sind auch hier rechnerisch und experimentel
ermittelte Fourier-Koeffizenten in Abbildung \ref{fig:np} eingezeichnet. Die experimentellen
Größen wurden wieder mit einem Fehler von 3\% versehen. Die Amplitude der rechnerischen Werte
für $ k = 10$ wurden wieder mit 10 Volt angenommen. Aber wie zu sehen ist passen die errechneten
Werte für $k = 10$ nicht zu unseren experimentellen Werten. Die Ampltitude A wurde für $k = 100$
mit 50 Volt angenommen. Hier sieht man sehr gut den sog. Dirac-Kamm der wiederum zu unseren
experimentellen Werten passt.
\begin{figure}
  \centering
  \includegraphics{Nadelimpulsplot.pdf}
  \caption{Rechnerische und experimentelle Fourier-Amplituden des Nadelimpulses}
  \label{fig:np}
\end{figure}
\FloatBarrier


\subsection{Zusammensetzung der Schwingunsformen aus ihren Komponenten}
