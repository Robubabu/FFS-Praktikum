\section{Theorie}
\label{sec:Theorie}

\subsection{Fouriersches Theorem}
\label{sec:Theorem}
Sei $f(t)$ periodisch in $t$ und steig, so lässt sich $f$ durch die sog. \textit{Fourier-Reihe} darstellen, welche wie folgt festgelegt ist:
\begin{equation}
  %F(t)= \frac{a_0}{2} + \sum_{n = 1}^\infty \Bigl a_n cos \bigl \frac{2 n \pi t}{T} \bigr + b_n sin \bigl \frac{2 n \pi t}{T} \bigr \Bigr
  \label{eqn:Fourier-Reihe}
\end{equation}
Wobei $T$ die Periodendauer bezeichnet und $a_n$ gegeben ist durch
\begin{equation}
  a_n = \frac{2}{T} \int_0^T f(t) cos(\frac{2 n \pi t}{T}) \symup{d}t .
  \label{eqn:an}
\end{equation}
$b_n$ ist analog durch
\begin{equation}
  b_n = \frac{2}{T} \int_0^T f(t) sin(\frac{2 n \pi t}{T}) \symup{d}t
  \label{eqn:bn}
\end{equation}
gegeben \cite{sample}.
Die Bestimmung dieser Koeffiezienten wird \textit{Fourier-Analyse} genannt und liefert ein diskretes Spektrum für alle periodischen Funktionen.

\subsection{Fourier-Transformation}
\label{sec:Trafo}
Mittels Fourier-Transformation $f(t) \to g(v)$ kann man das gesamte Frequenzspektrum $g(v)$ von $f(t)$ auf einmal berechnen, bzw. aus diesem $f(t)$. Sie ist gegeben durch
\begin{equation}
  g(v) = \int_{-\infty}^{\infty} f(t)*exp(ivt) \symup{d}t
\end{equation}
bzw
\begin{equation}
  f(t) = \frac{1}{2\pi} \int_{-\infty}^{\infty} g(v)*exp(-ivt) \symup{d}v.
\end{equation}
Bei Messungen des Spektrums ist zu beachten, dass selbst ein perfekt periodisches Signal zu einem kontinuierlichen Spektrum führen wird, da nicht von $-\infty$ bis $\infty$ integriert wird, sondern lediglich über einen begrenzten Zeitraum. Dies führt außerdem zu bedeutungslosen Nebenmaxima.
