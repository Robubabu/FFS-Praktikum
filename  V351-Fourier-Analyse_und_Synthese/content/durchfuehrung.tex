\section{Durchführung}
\label{sec:Durchführung}
\subsection{Messung des Frequenzspektrums}
\label{sec:Messung}
Im ersten Teil des Versuchs wird das Frequenzspektrum $g(v)$ gemäß \eqref{eq:trafo} bis zur 9. Oberwelle gemessen. Hierzu werden eine Rechteck- und eine Sägezahnspannung, sowie ein Nadelimpuls in ein digitales Oszilloskop eingespeißt, welches über die Einstellung \textit{FFA} selbstständig das Frequenzspektrum des Eingangssignals integriert. Hierbei ist zu beachten, die integrationsbedingten Nebenmaxima nicht mit abzulesen. Sie lassen sich an ihrem irregulären, willkürlichen Auftreten erkennen.

\subsection{Fouriersynthese}
\label{sec:Synthese}
Zur Fouriersynthese werden zuerst die nötigen Fourierkoeffizienten mit Hilfe der Formeln \eqref{eqn:an} \& \eqref{eqn:bn} berechnet. Nun werden an einem Oberwellengenerator 9 $9$ Oberwellen eingestellt und mitteneinander in Phase gebracht. Hierzu werden sie nacheinander mit der ersten zu Lissajous-Figuren überlagert. Anschließend wird ihre Amplitude möglichst exakt den berechneten Koeffizienten entsprechend eingestellt und alle Schwingungen werden addiert. Dies wird wie bereits in \ref{sec:Messung} für eine Rechteck- und eine Sägezahnspannung, sowie ein Nadelimpuls durchgeführt.
