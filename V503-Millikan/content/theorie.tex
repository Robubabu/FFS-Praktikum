\section{Zielsetzung}
Ziel des Versuches ist die Bestimmung der Elementarladung $e_0$.

\section{Theorie}
\label{sec:Theorie}
Beim Millikan-Versuch lässt sich die Ladung von Öltröpfchen über die sogenannte \textit{Schwebemethode} bestimmen. Die geladenen Tröpfchen werden hierbei in ein homogenes E-Feld variabler Stärke eingebracht und durch eine der Gewichtskraft entgegenwirkende Spannung zum Stillstand gebracht. Offensichtlich gilt durch das Kräftegleichgewicht
\begin{equation}
  \frac{4 \pi}{3} r_{korr} ^3 \rho_{Oel} g = q E \qquad \implies \qquad q = \frac{4 \pi}{3} \cdot\frac{r_{korr} ^3 \rho_{Oel} g}{E}.
  \label{eqn:q}
\end{equation}
Das Gewicht ist hierbei über die Dichte des verwendeten Öls und das Volumen der kugelförmigen Tropfen ausgedrückt. $q$ benennt die Ladung des Tropfens, welche ein Vielfaches der Elementarladung beträgt ($q = n \cdot e_0$, $n \in \mathbb{N}_0$), $E$ die elektrische Feldstärke.
Um den Radius $r_korr$ der Tropfen zu bestimmen, wird die Falleschwindigkeit $v_0$ ohne $E$-Feld gemessen und mit Hilfe der Stokesreibung in Luft (Viskosität $\eta_L$), des Druckes $p$, sowie der empirischen Korrekturgröße $B$ über
\begin{equation}
  r_{korr} = \sqrt{\left(\frac{\symup{B}}{2p}\right)^2 + \frac{9 \eta_L v_0}{2 g \rho_{Oel}}} -\left(\frac{\symup{B}}{2p}\right)
  \label{eqn:rkorr}
\end{equation}
verrechnet.
