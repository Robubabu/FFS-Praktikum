\section{Diskussion}
\label{sec:Diskussion}
\subsection{Relativer Fehler}
Alle relativen Fehler wurden nach der Formel
\begin{equation*}
  \tilde{x} = \frac{ \lvert x_{lit} - x_{mess} \rvert}{\lvert x_{lit} \rvert}
  \cdot 100 \%
\end{equation*}
berechnet, dabei bezeichnet $x_{lit}$ den Literaturwert der Messgröße $x_{mess}$.
\subsection{Zur Bestimmung der Elementarladung und Avogadro-Konstante}
Um ein quantitativen Vergleich durchzuführen zu können wird der relative
Fehler der bestimmten Größen zu ihren Literaturwerten \cite{scipy} bestimmt.
Für die bestimmte Elementarladung $q = \SI{1.424(5)e-19}{\coulomb}$ ergibt sich
mit Literaturwert $e_0 = \SI{1.6021766208e-19}{\coulomb}$, der relative Fehler
von \SI{11.1(3)}{\percent}. Für den bestimmten Wert für die Avogadro-Konstante
$N_A = \SI{6.78(2)e23}{\per\mol}$ und den Literaturwert \cite{scipy}
$ \symup{N}_A = \SI{6.022140857e+23}{\per\mol}$, ergibt sich ein relativer
Fehler von \SI{12.5(4)}{\percent}.
