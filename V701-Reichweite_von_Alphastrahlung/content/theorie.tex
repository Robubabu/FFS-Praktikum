\section{Ziel}
Das Versuchsziel besteht darin die Reichweite von $\alpha$-Strahlung zu bestimmen.
\section{Theorie}
\label{sec:Theorie}
Durch Bestimmung der Rechweite kann auch die Energie von $\alpha$-Strahlung
bestimmt werden. Die Energie der $\alpha$-Strahlung wird aber durch Anregund
und Zerfall sowie Ionisation von Molekühlen abnehmen,
elastische Stöße mit dem Medium (hier: Luft)
ist dabei ehr ein maginaler Faktor.
Der Energieverlust $\sfrac{-dE_{\alpha}}{dx}$ ist abhängig vom Druck $p$ des Mediums und der
Energie des $\alpha$-Teilchens.
Für große Energien wurde beschreibt die Bethe-Bloch-Gleichung \cite{Anleitung}
den Energieverlust wie folgt:
\begin{equation}
  -\frac{dE_{\alpha}}{dx} = \frac{z^2 e^4}{4 \pi \epsilon_0 m_e } \cdot \frac{nZ}{v^2} \cdot \ln{\left(\frac{2 m_e v^2}{I} \right)} \; .
  \label{eqn:BBG}
\end{equation}
Dabei bezeichnet $z$ die Ladung, $v$ die Geschwindigkeit der $\alpha$-Teilchen und
$Z$ die Ordnungszahl, $n$ die Teilchendichte und $I$ die Ionisierungsenergie des
Mediums. Die Beth-Bloch-Gleichung \eqref{eqn:BBG} verliert ihre Gültigkeit bei
kleinen Energien, da Ladungsaustauschprozesse statfinden. Deshalb wird die
mittlere Reichweite $R_m$ gemessen, diese beschreibt die Reichweite die noch von der
Hälfte aller $\alpha$-Teilchen erreicht wird, da die Austrittsenergie nicht für
alle Teilchen gleich ist.
Die mittle Reichweite $ R_m $ von $\alpha$-Strahlung mit $E_{\alpha} \leq \SI{2.5}{\mega\eV}$
in Luft kann durch die Gleichung
\begin{equation}
  R_m = 3,1 \cdot E_{\alpha} ^{\sfrac{3}{2}} \qquad \left(R_m \;\text{in}\; \si{\milli\meter} , E_{\alpha} \;\text{\alpha}\; \si{\mega\eV} \right)
  \label{eqn:Rm}
\end{equation}
beschrieben werden, diese Gleichung wurde empirisch Bestimmt.
Die Reichweite von $\alpha$-Teilchen in einem Gas ist proportional zum Druck $p$,
so lange wie Tempertur und Volumen konstant gehalten werden.
So gilt für einen festen Abstand $x_0$ zwischen Strahler und Detektor für die
effektive Länge $x$ der Zusammenhang
\begin{equation}
  x = x_0 \cdot \frac{p}{p_0}\; ,
  \label{eqn:x}
\end{equation}
dabei bezeichnet $p_0$ den Normaldruck \SI{1013}{\milli\bar}.
