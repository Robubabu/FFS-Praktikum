\section{Durchführung}
\label{sec:Durchführung}
\paragraph{Versuchsaufbau}
Die Versuchsapparatur ist wie in Abbildung \ref{fig:VB} dargestellt aufgebaut.
In den Glaszylinder in dem ein beliebiger konstanter Druck eingestellt werden kann
ist ein Halbleiter-Sperrschichtzähler fest verbaut und ein bewegliches Am-Preperat.
Bei einem einfallenden Ion auf den Zähler wird ein Strompuls registriert, indem im Halbleiter-Sperrschichtzähler Elektronenlochpaare gebildet werden, welche durch eine von außen anliegende Spannung zum Stromfluss führen.
Die Energie des Teilchens ist proportional zur Pulshöhe und wird in Form eines
Histogramms festgehalten. Die Energie wird hierbei aus der Stromstärke von einem Multi-Channel-Analyzer (MCA) oder Vielkanalanalysator ermittelt, indem er die ankommenden Stromsignale ihrer Stärke nach geordnet in Kanäle einsortiert und die Daten so digitalisiert. Dies wird dann von einem Computer dargestellt. Bevor die Messung beginnen kann sollten aber die Diskriminatorschwelle am Vielkanalanalysator
so eingestellt werden, dass das "Rauschen" von Umgebungseffekten verhindert wird. Der Diskriminator tut hierbei nichts weiter, als ab einer gewünschten unteren Schwelle alle Signale herraus zu filtern.

\begin{figure}
  \centering
  \includegraphics[height=7cm]{logos/Versuchsaufbau.png}
  \caption{Versuchsapparatur zur Bestimmung der Reichweite von \texorpdfstring{$\alpha$}{math}-Teilchen \cite{Anleitung}.}
  \label{fig:VB}
\end{figure}
\paragraph{Bestimmung der Reichweite von \texorpdfstring{$\alpha$}{math}-Teilchen}
\begin{itemize}
  \item Zuerst wird der Glaszylinder evakuiert und eine Position für das Präperat gewählt
  die während des ganzen Messprozesses konstant gehalten wird.
  \item Nun wird \SI{120}{\second} die Zählrate gemessen.
  \item Danach wird der Druck um \SI{50}{\milli\bar} erhöht und gemessen.
  \item Die ersten drei Schritte werden so lange wiederhohlt bis ein merklicher
  Abfall der Zählrate zu beobachten ist. Dann sollte die Erhöhung der Drucks in
  kleineren Schritten erfolgen, so lange bis die Zählrate wieder konstant bleibt.
\end{itemize}
\paragraph{Statistik des radioaktiven Zerfalls}
Nun werden Position und Druck konstant gehalten. Dann wird 200 mal die
Zählrate über \SI{10}{\second} gemessen.
