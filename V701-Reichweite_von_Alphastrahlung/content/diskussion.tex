\section{Diskussion}
\label{sec:Diskussion}

\subsection{Mittlere Reichweite}
Die Abweichung der Energiesteigungen der ersten und zweiten Messreihe ist mit $\Delta\frac{\symup{d}E_2}{\symup{d}x}= \SI{0.3 \pm 5.7}{\mega\eV\per\meter}\approx 0$ vernachlässigbar.
Die Reichweite der Strahlung liefert für beide Messreihen einen sehr geringen Wert von unter $2\si{\centi\meter}$, was sich mit dem Experiment deckt, da wir bei Normaldruck (ca 1000 mBar) bei 2.9 cm nur gerade eben noch vereinzelte Impulse messen konnten.

\subsection{Statistik des Zerfalls}
Die relative Abweichung der Standardweichung der Gauß funktion beträgt $\Delta \sigma \approx 5.4\%$. Die erwartete Poisson-Verteilung lies sich nicht fitten, es wären also längere Messreihen notwendig um den wahren statistischen Charakter des $\alpha$-Zerfalls zu erfassen.
