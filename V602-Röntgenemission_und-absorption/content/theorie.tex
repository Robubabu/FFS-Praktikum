
\paragraph{Zielsetzung}
In diesem Versuch soll das Emissionsspektrum einer Cu-Röntgenröhre und
das Absorberspektrum verschiedener Absorbermatrialien gemessen und ausgewertet
werden.

\section{Theorie}
\label{sec:Theorie}
Grundsätzlich werden bei diesem Versuch Elektronen beschleunigt und auf ein
Anodenmaterial geschossen. Dabei können dann im Bereich der gegebenen Vorraussetzungen
zwei Effekte auf treten, wobei sich ein kontinuierliches Spektrum und ein
charakteristisches messen lässt. Das kontinuierliche Bremsspektrum ist zu
beobachten, wenn die Elektronen abgebremst werden und dabei ihre kinetische
Energie als Lichtquant abgeben. 
