
\paragraph{Zielsetzung}
In diesem Versuch soll das Emissionsspektrum einer Cu-Röntgenröhre und
das Absorberspektrum verschiedener Absorbermatrialien gemessen und ausgewertet
werden.

\section{Theorie}
\label{sec:Theorie}
Grundsätzlich werden bei diesem Versuch Elektronen beschleunigt und auf ein
Anodenmaterial geschossen. Dabei können dann im Bereich der gegebenen Vorraussetzungen
zwei Effekte auftreten, wobei sich ein kontinuierliches Spektrum und ein
charakteristisches messen lässt. Das kontinuierliche Bremsspektrum ist zu
beobachten, wenn die Elektronen abgebremst werden und dabei ihre kinetische
Energie als Lichtquant abgeben. Das charakteristische Spektrum lässt sich beobachten,
wenn das Elektron ein Elektron aus einer Schale eines Atoms des Anodenmaterials
schießt, dort eine Leerstelle hinterlässt und ein Elektron aus einer äußeren Schale
nachrutscht und dabei ein Röntgenquant aussendet. Die Schalen werden dabei mit
$K,L,M,$ usw. bezeichnet. Sind diese mit einem Index mit griechischem Buchstaben versehen,
so kann man daraus erkennen aus welcher Schale das Elektron nachgerückt ist.
Die Energie des ausgesendeten Röntgenquants ist
\begin{equation}
  \symup{h} \nu = E_m - E_n
  \label{eqn:erq}
\end{equation}
dabei bezeichnen $E_m$ und $E_n$ die Energie der Energieniveaus.
Die Bindungsenergie $E_n$ der Elektronen auf der n-ten Schale ist
\begin{equation}
  E_n = -\symup{E_{ryd}} z_{eff}^2 \cdot \frac{1}{n^2} \; ,
  \label{eqn:be}
\end{equation}
dabei bezeichnet $E_{ryd} = 13,6 \si{\eV}$ die Rydbergenergie und
$z_{eff} = Z - \sigma $ die effektive Kernladung mit der Abschirmkonstante $\sigma$.
Diese ist von Nöten, da die Elektronen miteinander wechselwirken und so die
Kernladung abschirmen.
Der Comptoneffekt und der Photoeffekt sind die Effekte die bei der Messung von
Absorption von Röntgenstrahlung unter $ 1 \si{\mega \eV}$ am stärksten
hervortreten. Ist die Photonenenergie gerade größer als die Bindungsenergie eines
Elektrons so steigt der Absorptionskoeffizient, der sonst bei steigender Energie
abfällt, stark an und es ensteht eine sogenannte Absorptionskante. Die Lage einer
solchen Kante ist so gut wie identisch zu der Bidnungsenergie eines Elektrons.
Unter Zuhilfenahme der Braggschen Reflexion kann die Energie des einfallenden
Röntgenlichts an Hand der Braggschen Bedingung
\begin{equation}
  2 d \symup{sin} \theta = n \lambda
  \label{eqn:bragg}
\end{equation}
bestimmt werden.
