\section{Durchführung}
\label{sec:Durchführung}
\paragraph{Versuchsaufbau}
Der Versuchsaufbau besteht eigentlich nur aus einer Kupfer-Röntgenröhre,
einem LiF-Kristall (mit Gitterkonstante $d_{LiF} = 201,4 \si{\pico \meter}$)
und einem Geiger-Müllerzählrohr, die in einer festen Apparatur
verbaut sind. Die Apparatur wird durch einen Computer gesteuert und die Messwerte
werden direkt digitalisiert. So kann über den Computer die Messart, der Drehmodus,
der Kristallwinkel und die Integrationszeit eingestellt werden.
Bei allen Messungen wird die Röntgenröhre mit einer Beschleunigungsspannung von
$U_B = 35 \si{\kilo \volt} $ und einem Emissionsstrom $I = 1 \si{\milli \ampere}$
betrieben.

\paragraph{Überprüfung der Bragg Bedingung}
Zur Überprüfung der Bragg Bedinung wird der Kristallwinkel $ \theta = 14 \si{\degree}$
gesetzt und das Zählrohr fährt den Winkelbereich von $24 \si{\degree}$ bis
$ 30 \si{\degree} $ ab. Dabei soll bei jedem Winkelunterschied von $ 0,1 \si{\degree}$
$20 \si{\second}$ gemessen werden.

\paragraph{Messung des Emissionsspektrums einer Kuper-Röntgenröhre}
Im Messprogramm des Computers soll nun der 2:1 Kopplungsmodus gewählt werden und
der Kristall soll den Winkelbereich von $4 \si{\degree} $ bis $26 \si{\degree}$
abfahren, wobei dabei soll das Zählrohr in $0,2 \si{\degree}$-Schritten fortläuft. In jedem Schritt wird $5\si{\second}$gemessen.

\paragraph{Messung des Absorptionsspektrum verschiedener Metalle}
Das Messprogramm ist in soweit gleich wie zuvor bis auf die Tatsache, dass
der abgefahrene Kristallwinkel nun vom Erwartungswert abhängt und das Zählrohr
in $0,1 \si{\degree}$-Schritten messen soll in den $ 20 \si{\second}$ gemessen wird.
Zudem wird vor das Zählrohr der Absorber geschraubt.
