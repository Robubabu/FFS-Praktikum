\section{Durchführung}
\label{sec:Durchführung}

\paragraph{Integrale Energieverteilung der Elektronen}
Die integrale Energieverteilung wird einmal bei ca. $\SI{296}{\kelvin}$ und einmal bei ca. $\SI{432}{\kelvin}$ durchgeführt. Hierzu wird $U_B = 11 \si{\volt}$ fest eingestellt und $U_a$ gleichmäßig von $0\si{\volt}$ auf ca. $11\si{\volt}$ erhöht. Mit einem X-Y-Schreiber wird $I_a$ gegen $U_a$ aufgezeichnet.

\paragraph{Franck-Hertz-Kurve}
Die charakteristischen Franck-Hertz-Kurven werden bei unterschiedlichen Temperaturen und einer Bremspannung von $U_a = 1\si{\volt}$ aufgezeichnet, da die Temperatur (über Gasdruck und $\Delta E$) einen großen Einfluss auf die Güte der Messung hat. $U_b$ wird zwischen $0$ und $60 \si{\volt}$ varriert.

\paragraph{Ionisierungsenergie des Quecksilberatoms}
Die Ionisierungsenergie der Quecksilberatome wird bei $T \approx 376 \si{\kelvin}$ bestimmt, indem bei $U_a = -30 \si{\volt}$ $I_a$ gegen $U_b$ aufgetragen wird.
