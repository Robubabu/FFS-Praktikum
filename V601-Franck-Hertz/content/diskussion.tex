\section{Diskussion}
\label{sec:Diskussion}
\subsection{Methoden}
Alle relativen Fehler wurden nach der Formel
\begin{equation*}
  \tilde{x} = \frac{ \lvert x_{lit} - x_{mess} \rvert}{\lvert x_{lit} \rvert}
  \cdot 100 \%
\end{equation*}
berechnet, dabei bezeichnet $x_{lit}$ den Literaturwert der Messgröße $x_{mess}$.
\subsection{Zur den Energieverteilungen}
Zuerst ist zu erwähnen, dass die integralen Energieverteilungen wie erwartet aussehen.
Bei Abbildung \ref{fig:8a1} ist zu sehen, dass der Graph relativ spät abflacht im
Vergleich zum Graphen von Abbildung \ref{fig:8a2}. Dies ist klar durch den
größeren Dampfdruck zu erklären, da somit auch die Trefferwahrscheinlichkeit der
Elektronen sich erhöht was dazu führt, dass die Elektronen schneller abgebremst
werden und somit der Graph schneller abflacht.
\subsection{Zur Franck-Hertz-Kurve}
Der relative Fehler bezüglich des Literaturwertes \cite{wiki:fh} für die Anregungserngie
$U_{1,lit} = 4,9 \si{\eV} $ liegt bei $ \left(3 \pm 4 \right) \si{\percent} $.
Genauso für den Literaturwert \cite{wiki:fh} der Wellenlänge $ \lambda_{lit} = 253 \si{\nano \meter} $
auch hier liegt der relative Fehler bei $ \left(3 \pm 4 \right) \si{\percent} $.
Die Abweichungen können duch den Einfluss des Dampfdruckes, des Energie-Spektrums der Elektronen
und des Kontaktpotentials erklärt werden.
\subsection{Zur Ionisierungsspannung}
Die Ionisierungsspannung hat im Anstieg den (wie in der Anleitung \cite{Anleitung}) beschrieben
Verlauf. Dennoch wird nach erreichen des Maximums ein starker Abfall verzeichnet.
Dieser ist durch eine Gasentladung zu erklären was zu einem Abfall des Auffängerstroms führt.
