\section{Diskussion}
\label{sec:Diskussion}
Im folgenden wurden die relativen Fehler der Schallgeschwindigkeiten betrachtet,
um die Abweichungen der Messwerte diskutieren zu können. Der relative Fehler wurde,
wie in der Auswertung (s.Kapitel \ref{sec:Auswertung}) geschildert, berechnet.
Der Literaturwert der
Schallgeschwindigkeiten wurde aus SciPy \cite{scipy} entnommen für 15 Celsius bei
einem atm.
Abweichungen der Messwerte für $c$ lassen sich durch
den Temperaturunterschied zum Literaturwert und die damit einhergehende
Betragsänderungen der Schallgeschwindigkeiten, sowie allgemeine Messunsicherheiten
erklären.
\subsection{Zur Wellenlänge \texorpdfstring{$\lambda$}{[math]}}
%-0.154+/-0.007
Dann ergibt sich bei der Messung der Wellenlänge der relative Fehler
\begin{equation*}
  \increment c = (15,4 \pm 0,7)\;\%.
\end{equation*}
Da der relative Fehler nicht auffallend groß ist, spricht der relative Fehler
für die Qualität der Messung.
\subsection{Zur Frequenzänderung des bewegten Senders}
%[0.04080662525677029+/-0.0071759188529015435]
Hier beträgt der relative Fehler
\begin{equation*}
    \increment c = (4,08 \pm 0,72)\;\%.
\end{equation*}
Auch hier ist die Abweichung vom Literaturwert nicht auffallend groß und macht
somit eine Aussage über die Qualität der Messung.
\subsection{Zur Frequenzänderung des bewegten Reflektors}
%[0.02189325545069723+/-0.00317546563311801]
%[0.03732348917366589+/-0.03233621106426434]
In der Auswertung dieses Versuchsteil wurden die Messwerte für  jede
Richtung einzeln gefittet. Daraus ergeben sich dann für die Messung in der
der Reflektor zur Schallquelle hinfährt der relative Fehler
\begin{equation*}
    \increment c = (2,19 \pm 0,32)\;\%
\end{equation*}
und für die Messung in der der Reflektor von der Quelle wegfährt der relative
Fehler
\begin{equation*}
    \increment c = (3,74 \pm 3,23)\;\%.
\end{equation*}
Die relative Fehler sind nicht besonders auffallend und somit Maß für die
Qualität der Messungen.
