\section{Theorie}
\label{sec:Theorie}

% Das als Doppler-Effekt bekannte Phänomen der geschwindigkeitsabhängigen
% Frequenzwahrnehmung äußert sich darin, dass, wenn man sich in Wellenberge
% "hinein" bewegt, bzw. sich von ihnen entfernt, die Frequenz einer Welle steigt
% bzw sinkt, da sich die relative Ausbreitungsgeschwindigkeit der Welle zum
% Betrachter bei fester Wellenlänge um die Relativgeschwindigkeit zwischen
% Betrachter und Quelle der Welle ändert.
Der Doppler-Effekt bezeichnet das Phänomen, dass sich die wahrgenommen
Frequenz in Abhängigkeit der Relativgeschwindigkeit zwischen dem Sender und dem
Empfänger ändert. Dabei ist zu bemerken, dass sich die Frequenz erhöht, wenn der
Empfänger sich auf den Sender zu bewegt und sinkt, falls der Empfänger sich
von dem Sender wegbewegt.
Bezeichnet $\nu_{\symup{E}}$ die am Empfänger auftretende Frequenz, $\nu_0$ die
an der Quelle abgegebene Frequenz, $\lambda_0$ die Wellenlänge und v die
Relativgeschwindigkeit zwischen Sender und Empfänger, so ergibt sich die
Formel des Doppler-Effekts zu
\begin{equation}
  \nu_{\symup{E}} = \nu_0 + \frac{\symup{v}}{\lambda_0}
  \label{eqn:nuE1}
\end{equation}
bzw. mit $\symup{c} = \nu_0 \lambda_0$ zu
\begin{equation}
  \nu_{\symup{E}} = \nu_0 \left(1+ \frac{\symup{v}}{\symup{c}}\right).
  \label{eqn:nuE2}
\end{equation}
Bewegt sich die Quelle jedoch mit v relativ zu dem Medium welches die Welle
trägt, so ändert sich die Wellenlänge scheinbar um
\begin{equation*}
  \Delta \lambda = \frac{\symup{v}}{\nu_0}.
\end{equation*}
Dies sorgt dafür, dass die Frequenz beim Empfänger
\begin{equation}
  \nu_{\symup{Q}} = \nu_0 \frac{1}{1-\frac{\symup{v}}{\symup{c}}}
  \label{eqn:nuQ}
\end{equation}
beträgt. Allerdings wird der Unterschied zwischen \eqref{eqn:nuE2} und
\eqref{eqn:nuQ} infinitisimal gering für niedrige Geschwindigkeiten ($<< c$).
